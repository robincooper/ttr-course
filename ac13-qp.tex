\documentclass[a4wide]{article}

%\usepackage{natbib}
%\bibpunct{(}{)}{,}{a}{}{;}
%\usepackage{linguex} 
\usepackage{avm,lingmacros}%,jg-gen} 
\usepackage{url}

%----- MARGINS ----------------

\setlength{\paperwidth}{494pt} % A4
\setlength{\textwidth}{450pt}
% \setlength{\hoffset}{6pt}
\setlength{\oddsidemargin}{0pt}
\setlength{\paperheight}{846pt} % A4
\setlength{\textheight}{650pt}
% \setlength{\voffset}{0pt}
\setlength{\topmargin}{-10pt}
% \setlength{\headheight}{0pt}
% \setlength{\headsep}{-10pt}
%\setlength{\parindent}{0pt}
%\setlength{\parskip}{4pt plus 2pt minus 1pt}
% \newcommand{\record}[1]{$\left[\mbox{\begin{tabular}{lcl} #1
% \end{tabular}}\right]$} 
% \newcommand{\smallrecord}[1]{$\left[\mbox{\begin{tabular}{@{}l@{}c@{}l@{}} #1
%  \end{tabular}}\right]$}
% \newcommand{\field}[2]{#1 & = & #2}
% \newcommand{\tfield}[2]{#1 & : & #2}
% \newcommand{\smalltfield}[2]{#1:#2 & &}
% \newcommand{\mfield}[3]{#1=#2 & : & #3}
% \newcommand{\smallmfield}[3]{#1=#2:#3 & &}
% \newcommand{\hfield}[2]{{\sc #1} & & #2}

%------------------------------

\newcommand{\ignore}[1]{}
\newcommand{\w}[1]{`\emph{#1}'}

\title{}
\author{}
\date{}

%Records

\newcommand{\record}[1]{$\left[\mbox{\begin{tabular}{lcl} #1
\end{tabular}}\right]$} 
\newcommand{\smallrecord}[1]{$\left[\mbox{\begin{tabular}{@{}l@{}c@{}l@{}} #1
\end{tabular}}\right]$}

\newcommand{\field}[2]{#1 & = & #2}
\newcommand{\tfield}[2]{#1 & : & #2}
\newcommand{\smalltfield}[2]{#1:#2 & &}
\newcommand{\mfield}[3]{#1=#2 & : & #3}
\newcommand{\smallmfield}[3]{#1=#2:#3 & &}
\newcommand{\hfield}[2]{{\sc #1} & & #2}

\begin{document}
\pagestyle{empty}

%--------- title --------------------------------------------------
\vspace*{-8\baselineskip}

\begin{center}
{\bf \Large 
%On the (Non)Type Unity of Questions and Propositions
On Distinguishing Questions and Propositions}
\end{center}
%------------------------------------------------------------------
\vspace*{5pt}

\ignore{ Checklist:

\begin{itemize} 
 
\item Questions \textbf{can} be conjoined and disjoined with other questions 
 
\item Questions \textbf{cannot} be conjoined or disjoined with
  propositions

\item Speech acts involving queries \textbf{can} be conjoined and
  disjoined with other queries

\item Speech acts involving assertions of propositions and queries
  \textbf{can} be conjoined and disjoined in that order

\item Questions \textbf{cannot} be conditionalized (although questions
  based on conditional propositions can be constructed)

\item Speech acts which are queries \textbf{can} be conditionalized

\item Questions \textbf{cannot} be negated (although questions based
  on negative propositions can be constructed)

\ignore{\item Speech acts which are queries (or any other speech acts)
  \textbf{cannot} be negated.  That is, \textit{Isn't he quite smart?}
  cannot mean ``I'm not asking you whether he is smart''.
 }
\end{itemize} 
}

%JG comment: replace modality or add negation so we can bring it in the picture

Does analyzing questions require entities distinct from propositions?
Both Hamblin and Karttunen gave arguments for distinguishing questions
as an ontological category from propositions---([5]\ignore{\cite{hamblin58}})
pointing out that interrogatives lack truth values (\textit{It's true/false who came yesterday}), to which one
can add their incompatibility with a wider scoping alethic modality
(\textit{\# Necessarily, who will leave tomorrow?}) whereas ([6]\ignore{\cite{kartt77}})
pointed to the existence of predicates that select interrogatives, but
not for declaratives and vice
versa:
\textit{Bo asked/investigated/wondered/\# believed /\# claimed who came
yesterday},  \textit{Bo \# asked/\# investigated/\# wondered/ believed
/claimed that Mary came yesterday}.

%JG: need to cite also Francez and Nelken, Shan and Nelken L\&P papers
%to show this is a trend not limited to InqSem has been developing an influential account which treats propositions andquestions as entities belonging to a single ontological category,
%which in particular allows for entities that are both propositions
%and questions (hybrids, that are both assertions and
%inquisitive.). An important motivation for this is the desire to allow for an elegantlogic which offers a simple treatment of Boolean operations.

Recently there have been a number of proposals that questions and propositions are of
a single ontological category% , which offers a simple treatment of Boolean operations
(see [7,8]\ignore{\cite{nelken-francez,nelken-shan06}}) and most influentially work
in Inquisitive Semantics (IS) ([4]\ignore{\cite{groenendijk09}}).  A significant
argument for this is examples like \textit{If Kim is not available,
  who should we ask to give the talk?} where propositions and
questions can apparently be combined by boolean connectives.

In this paper we will consider potential problems for this as a
strategy as an analysis for natural language. We argue that although
speech acts involving questions
and propositions can be combined by boolean connectives they are not closed under boolean operations.  Furthermore
we argue that the propositions and questions  \textit{qua} semantic
objects cannot
be combined by boolean operations at all. This, together with
the examples above, strongly suggests that questions and propositions are distinct
types of semantic objects. We give an account of
the distinction in TTR (Type Theory with Records,
[2]\ignore{\cite{cooper-ddl}}).

% proposition and it has been noted in the literature that a sentence
% like \textit{The manager knows that John's smart and what
%     qualifications he has} seems acceptable.  However, we believe that
%   the only way of interpreting this is one

We use embedding under attitude verbs as a test for propositions and
questions as semantic objects.  Here we do not find mixed
boolean combinations of questions and propositions.  Thus, for
example, \textit{wonder} selects for an embedded question and
\textit{believe} for an embedded proposition but a mixed conjunction
does not work with either, showing that it is neither a question nor a proposition: \textit{The manager *wonders/*believes that several  people left and what rooms we need to
clean}.  The verb \textit{know} is compatible with both interrogative
and declarative complements, though ([9,3]) \ignore{\cite{vendler72,gs00}} argue that such predicates do not take
questions or propositions as genuine arguments (i.e. not purely
referentially), but involve {\it coercions}   which leads to a predication of a {\it fact}.
This leads to the expectation that sentences involving decl/int conjunctions such as \textit{The manager knows that John's smart and what    qualifications he has} can only be understood
where the verb is  distributed over the two conjuncts: ``knows that John's smart and
  knows what qualifications he has''.  Compare \textit{It's surprising
    that the conference was held at the usual time and so few people
    registered} and \textit{It's surprising that the conference was
    held at the usual time and how few people registered}.  In the
  second mixed case there is only a reading which entails that it is
  surprising the conference was held at the usual time whereas
  arguably in the first sentence \ignore{(if it is acceptable)} only the conjunction but not the
  individual conjuncts need be surprising. 
Embedded conditional questions are
  impossible: \textit{*The manager wonders if Hollande left, whether we need to
    clean the west wing.}, although, of course, embedded questions
  containing conditionals are fine:  \textit{The manager wonders whether, if Hollande left, we need to
    clean the west wing.}

%(and we do not yet have a precise theory of speech acts which would account for it)

Why, then, do apparent mixed boolean combinations appear in root
sentences?  Our answer
is that natural language connectives, in addition to their
function as logical connectives combining propositions, can be used to
combine speech acts into another single speech act.  This, however, can only be expressed in root
sentences and speech acts are not closed under operations
corresponding to boolean connectives.  For
example, \textit{John's very smart but does he have any
  qualifications?} where a query follows an assertion is fine whereas
the combination of an assertion with a preceding query is not:
\textit{*Does John have any qualifications and/but he's smart} is not.
This is puzzling because a discourse corresponding to a string of the
same separate speech acts works well:  \textit{Does John have any
  qualifications? (no answer) But he's smart.}.  Similarly, while we
can apparently conditionalize a query with a proposition
(\textit{If Hollande left, do we need to clean the west wing},
i.e. ``If Hollande left, I ask you whether we need to clean the west
wing''), we cannot conditionalize an assertion with a question
(\textit{*If whether Hollande left/did Hollande leave, we need to
  clean the west wing}) and neither can we conditionalize a query
with a question (\textit{*If who left, do we need to clean the west
  wing?}).  However we treat these facts, it seems
clear that it would be dangerous to collapse questions and
propositions into the same type of semantic object and allow general
application of semantic boolean operators.  This would seem to force
you into a situation where you have to predict acceptability of these
sentences purely on the basis of a theory of syntax, although
semantically/pragmatically they would have made perfect sense.  It
seems to us that distinguishing between questions and propositions and
combinations of speech acts offers a more explanatory approach.

% We start at the speech act level: whereas queries can be con/disjoined
% in any order (wh with polar and vice versa), speech acts involving
% assertions of propositions and queries \textbf{can} be conjoined and
% disjoined only in that order:

% \noindent (2a) John's  very smart *and/but does he have any qualifications?\\
% (2b) Does John have any qualifications *and/*but he's smart(./?)\\
% (2c)  *What qualifications does John have or he's smart?

%  With conditionals most decl/int combinations are impossible, apart
%  from decl $\rightarrow$ int, which we will argue needs to be treated
%  as a queried conditional. Speech acts which are queries \textbf{can}
%  be conditionalized:

% \noindent (3a)*If who left, we need to clean the west wing  (wh$\rightarrow$decl)\\
% (3b) *If whether anybody left, we need to clean the west
%     wing (polar$\rightarrow$decl)\\
% (3c)*If who left, do we need to clean the west wing
%   (wh$\rightarrow$polar)\\
% (3d) *If whether anybody left, which rooms do we need to clean (polar$\rightarrow$decl)\\
% (3d) If Hollande left, I ask you whether we need to clean the west
%   wing
% %I ask you whether, if Hollande left we need to clean the west wing
 

%  It is important to tease apart distinctions boolean combinations of
%  speech acts and boolean combinations of questions or
%  propositions. Difficult to tease these apart in direct questions
%  (root interrogatives), embedded questions will help us, because
%  there the speech act reading is not available:

% \noindent (4a)Who left and what rooms do we need to clean?\\
% (4b) ``I ask you who left and what rooms we need to clean'' (question conjunction)\\
% (4c)``I ask you who left and I ask you what rooms we need to clean''
% (speech act conjunction)

% Whereas verbs like `wonder' select for questions and `believe' for propositions, 
%  factives/resolutives like `know' and `discover' are compatible with
%  interrogatives and declaratives, but there is extensive data
%  \cite{vendler72,gs00} that such predicates do not take
%  questions or propositions as genuine arguments (i.e. not purely
%  referentially). In both cases there is a {\it coercion} going on
%  which leads to a predication of a {\it fact}.

% On the basis of this, we see that con/dis-joined questions are questions. 
%  However, mixed con/dis-junctions are neither questions nor
%  propositions, with  and/or/but:

% \noindent (5a)  The manager knows/wonders/*believes who left and/or what
%     rooms we need to clean\\
% (5b) The manager knows/wonders/*believes whether anybody left
%     and/or what rooms we need to clean\\
% (5c)The manager ?knows/*wonders/*believes that several  people left and what rooms we need to
% clean (cf. The manager knows that several people left and
%   knows what rooms we need to clean)\\
% (5d) *The manager knows that John's smart but what
%     qualifications he has (cf.  *The manager knows that John's smart but knows
%   what qualifications he has.)

% Moreover, there are no conditional questions: 
 
% (6a) The manager wonders whether, if Hollande left, we need to
%     clean the west wing.\\
% (6b) *The manager wonders if Hollande left, whether we need to
%     clean the west wing.


In the formal development of the paper we will present a theory of
Austinian propositions including conjunction, disjunction and negation.
The type of Austinian propositions is the record type
(1a),where the type \textit{RecType}$^\dagger$ is a basic type
which denotes the type of (\textit{non-dependent}) record types closed under meet, join and
negation.  Truth conditions for Austinian propositions are defined in (1b).

\noindent (1a) \
\textit{Prop} =$_{def}$
\record{\tfield{sit}{\textit{Rec}}\\ \tfield{sit-type}{\textit{RecType}$^\dagger$}} \\
(1b) A proposition $p$ = \record{\field{sit}{$s_0$}\\
  \field{sit-type}{$ST_0$}} is true iff $s_0 : ST_0$

If $p$:\textit{Prop} and $p$.sit-type is $T_1\wedge T_2$ ($T_1\vee
T_2$, $\neg T$) we say that $p$ is the conjunction (disjunction) of
\record{\field{sit}{$p$.sit} \\
  \field{sit-type}{$T_1$}} and  \record{\field{sit}{$p$.sit} \\
  \field{sit-type}{$T_2$}} (the negation of \record{\field{sit}{$p$.sit} \\
  \field{sit-type}{$T$}}).  

We also introduce a notion of \textit{Austinian questions} defined as records
containing a
record and a \textit{dependent} record type:

\noindent {\small (2a) \textit{Who} = \begin{avm}\[x : \textit{Ind}\\ 
c1 : person(x)\]\end{avm}; \textit{Whether} = {\it Rec};
(2b) `Who runs' $\mapsto$ \begin{avm}\[sit =$r_1$\\
dep-type = $\lambda r$:\textit{Who}(\[c : run($r$.x)\])\]\end{avm}; \\
(2c) `Whether Bo runs' $\mapsto$ \begin{avm}\[sit =$r_1$\\
dep-type = $\lambda r$:\textit{Whether}(\[c : run(b)\])\]\end{avm} }

Given this, we define the following relation between a situation and a
dependent type, which is the basis for defining key coherence
answerhood notions such as resolvedness and aboutness (weak partial
answerhood, [3]\ignore{\cite{gs00}}) and question dependence (cf.~erotetic implication, [10]\ignore{\cite{wisniewski-la}}):

$s$ \textit{resolves} $q$, where $q$ is $\lambda r:T_1(T_2)$, (in
  symbols $s ? q$)
  iff \textbf{either}\\ 
\hspace*{5em} (i) for some $a:T_1$  $s : q(a)$, \textbf{or} \\
\hspace*{5em}(ii) $a:T_1$ implies  $s : \neg q(a)$


Austinian questions can be conjoined and disjoined though not
negated.  (We argue that ``negative questions'' involve questions
relating to negative propositions rather than negations of positive
questions.  Such negative questions are crucially distinct from the
corresponding positive question, following [1]\ignore{\cite{cooper2012negative}}). The definition for conj/disj-unction, from which it follows that
$q_1$ and/or $q_2$ is resolved iff $q_1$ is resolved and/or $q_2$ is
resolved, is as follows:

 \record{\field{sit}{$s$} \\
        \field{dep-type}{     $\lambda r:T_1\
  (T_2)$       }} $\wedge$/$\vee$ \record{\field{sit}{$s$} \\
                                                  \field{dep-type}{   $\lambda r:T_3\ (T_4)$  }}
                                                = \\ 
\hspace*{5em}\record{\field{sit}{$s$} \\
                                                          \field{dep-type}{   $\lambda r$:\smallrecord{\smalltfield{left}{$T_1$} \\
                         \smalltfield{right}{$T_3$}}($q_1(r.\mathrm{left})\wedge /\vee q_2(r.\mathrm{right})$)    }}

In the full version of the paper we will show how the mixed
cases involving conjunctions of assertions and queries can be captured in a
QUD--based dialogue semantics using an algebraic approach to speech
events: whereas `and' indicates that the following question $q1$ is
independent of the current  Max(imal element of)QUD; `but'
indicates that $q1$   is not
independent, but unexpected given MaxQUD, whereas `or'  presupposes
the existence of an issue   that both $q1$ 
and   MaxQUD address, hence both are retained as MaxQUD. 

%JG: or mention negation?
\vspace{5pt}
{\small
\noindent [1] Cooper, R., Ginzburg, J. (2012): `Negative inquisitiveness and alternatives-based
  negation'. In: Aloni, M., Kimmelman, V., Roelofsen, F., Sassoon, G., Schulz,
  K., Westera, M. (eds.) {\it Logic, Language and Meaning, Lecture Notes in Computer
  Science}, vol. 7218, pp. 32--41. Springer Berlin Heidelberg 

\noindent [2] Cooper, R.  (2012): `Type theory and semantics in flux'. In: Kempson, R., Asher, N.,
  Fernando, T. (eds.) {\it Handbook of the Philosophy of Science, vol. 14:
  Philosophy of Linguistics}. Elsevier, Amsterdam

\noindent [3] Ginzburg, J., Sag, I.A. (2000):{\it  Interrogative Investigations: the form, meaning and
  use of English Interrogatives.} No. 123 in CSLI Lecture Notes, CSLI
  Publications, Stanford: California 

\noindent [4] Groenendijk, J., Roelofsen, F  (2009).: `Inquisitive semantics and pragmatics'. In:
  {\it Meaning, Content, and Argument: Proceedings of the ILCLI International
  Workshop on Semantics, Pragmatics, and Rhetoric}.

\noindent [5] Hamblin, C.L.  (1958): `Questions'. {\it Australian Journal of Philosophy}  36,  159--168
 
\noindent [6] Karttunen, L. (1977): Syntax and semantics of questions. Linguistics and Philosophy
  1,  3--44 

\noindent [7] Nelken, R., Francez, N.  (2002): `Bilattices and the semantics of natural language
  questions'. {\it Linguistics and Philosophy}  25,  37--64

\noindent [8] Nelken, R., Shan, C.C. (2006): `A modal interpretation of the logic of interrogation'.
 {\it  Journal of Logic, Language, and Information}  15,  251--271 

\noindent [9] Vendler, Z. (1972): {\it Res Cogitans}. Cornell University
Press, Ithaca 

\noindent [10] Wi{\'{s}}niewski, A.  (2001): Questions and
inferences. {\it Logique et Analyse}  173,  5--43
} 

%\end{thebibliography}

%\vspace*{-\baselineskip}
%{\tiny
%\bibliographystyle{splncs03}
%\bibliography{newest-jg-fin}
%}
\end{document}

