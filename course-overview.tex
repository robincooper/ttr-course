\documentclass[12pt]{article}

\title{Course Overview/Notes}
\author{}
\date{\today}


\begin{document}
\maketitle

\begin{verbatim}

Lecture 1: 

fix toc file

underline what pay offs we have

1. promise to analyze the printer conversation; 
examples of interaction--oriented words and constructions;

2. Reference to TIS where stuff is detailed. 

1.5 TTR: TTR as relating semantic ontology, grammatical ontology, 
and interaction

2. promise to say sthing about genres, disfls; quotation at least make
the reference?
3. benchmarks for NSUs and CRs; 
4. child language data; 
5. negation, quantification, anaphora; 
6. detailed theory of relevance and irrelevance.

The need to revise basic semantic assumptions: 
meaning, content r'ship;  (modifed in lecture 4; maybe 
also in lecture 2?)
equal access; (used as motivation for each CP having own DGB)
perfect communication; (modified in utterance processing protocol)
weak montogovianism (modified in utterance processing protocol)
Turn Taking Puzzle (explicated in lecture 4)

part 2: Basics of TTR

Lecture 2: TTR as a theory of semantic and grammatical ontology
event structure
frames (discharge initial obligation re TTR concerning sem ontology)
To what extent is there compatability between the view of grammar in
lecture 2 and that sketched in lecture 4 (sketched possibly too 
unsystematically)?
Where are the grammar rules that are utilized? (very partially
discharge 
initial obligation re TTR concerning gram ontology)

INCONSISTENCY between Cooper and Ginzburg on the grammar front

Lecture 3:
introduce propositions, questions, and maybe outcomes (rushed)!  
(discharge obligation re TTR concerning sem ontology)
introduce basics of interaction (rushed)! (initial discharging of 
negation (hereby discharging 5)

Lecture 4:
part 1 addresses 6, 3

part 2 addresses 1.5 (grammar) and 5 (quantification; a bit, no CR
stuff)

Lecture 5: (to be done by Fri PM)
NSUs: addresses 3.

Leaves 0.3- 0.5 lecture free: discuss what: genres? disfluencies,
multilogue?
Stuff necessary to analyze the original ``printer example'' from
lecture 1 (needs stuff on disfluencies and multilogue).
 
\end{verbatim}

\end{document}