\documentclass{beamer}

%\usepackage{qtree}

\usepackage{beamerthemetree}
\usepackage{color}
\setbeamertemplate{footline}[frame number]
\usepackage{graphicx}
%\usepackage[swedish]{babel} 
\usepackage[latin1]{inputenc}
\usepackage{natbib} 
%\renewcommand{\newblock}{}


\title{Negation}
\author{Jonathan Ginzburg\\
Universit\'e Paris-Diderot, Sorbonne Paris-Cit\'e\\
Robin Cooper\\ 
University of Gothenburg}
\date{An Introduction to Semantics using Type Theory with Records \\ ESSLLI
2012 \\ Lecture 3,  part 2}

\AtBeginSection[]
{
 \begin{frame}[plain]
   \frametitle{Outline}
 \tableofcontents[currentsection]
 \end{frame}
}

\newcommand{\backgroundyellow}{\beamertemplateshadingbackground{yellow!40}{magenta!20}}
\newcommand{\backgroundwhite}{\beamertemplateshadingbackground{white!100}{white!100}}

\newcommand{\ignore}[1]{}

\newenvironment{display}{\begin{center}}{\end{center}}%

\usepackage{times}
\usepackage{latexsym}
\usepackage{amsmath}
\usepackage{multirow}
\usepackage{url}

\usepackage{lingmacros}
\usepackage{avm}
\usepackage[latin1]{inputenc} 
 


\DeclareMathOperator*{\argmax}{arg\,max}
%\setlength\titlebox{6.5cm}    % Expanding the titlebox

% \title{Negation in dialogue}

% \author{Robin Cooper \\
% Department of Philosophy, Linguistics \\ and Theory of Science \\
% University of Gothenburg \\
% Box 200 \\ 405 30 G�teborg, Sweden \\
% {\tt cooper@ling.gu.se}
% \And
% Jonathan Ginzburg \\
% Univ. Paris Diderot, Sorbonne Paris Cit\'e\\
% CLILLAC-ARP (EA 3967)\\
% 75004 Paris, France\\
% {\tt yonatan.ginzburg@univ-paris-diderot.fr} }
 
% \date{\today}
\newcommand{\mq}{{\sc max-qud }}
%\newcommand{\ignore}[1]{}
\newcommand{\ben}{\begin{enumerate}}
\newcommand{\een}{\end{enumerate}}
\newcommand{\bqt}{\begin{quote}}
\newcommand{\eqt}{\end{quote}}
\newcommand{\bit}{\begin{itemize}}
\newcommand{\eit}{\end{itemize}}
\newcommand{\bt}{\begin{tabbing}}
\newcommand{\et}{\end{tabbing}}
\newcommand{\bd}{\begin{description}}
\newcommand{\ed}{\end{description}}
\newcommand{\bv}{\begin{verbatim}}
\newcommand{\ev}{\end{verbatim}}
\newcommand{\ba}{\begin{avm}}
\newcommand{\ea}{\end{avm}}
\newcommand{\ugh}{\#}
\newcommand{\whp}{{\it wh}-phrase }
\newcommand{\eeen}{\eenumsentence}

%Records

\newcommand{\record}[1]{$\left[\mbox{\begin{tabular}{lcl} #1
\end{tabular}}\right]$} 
\newcommand{\smallrecord}[1]{$\left[\mbox{\begin{tabular}{@{}l@{}c@{}l@{}} #1
\end{tabular}}\right]$}

\newcommand{\field}[2]{#1 & = & #2}
\newcommand{\tfield}[2]{#1 & : & #2}
\newcommand{\smalltfield}[2]{#1:#2 & &}
\newcommand{\mfield}[3]{#1=#2 & : & #3}
\newcommand{\smallmfield}[3]{#1=#2:#3 & &}
\newcommand{\hfield}[2]{{\sc #1} & & #2}

%Types
\newcommand{\down}[1]{[\ \!\check{}\ \!#1]}


\newcommand{\comment}[1]{\textsf{#1}}

\begin{document}
\frame[plain]{\titlepage}
\frame[plain]{\frametitle{Outline}\tableofcontents}

\frame{

\frametitle{References}

\cite{CooperGinzburg2011,CooperGinzburg2011a}

}

\section{Negation in dialogue}

\frame{

\frametitle{Classical view of negation}

\begin{itemize} 
 
\item truth functional connective 
 
\item true $\rightarrow$ false,  false $\rightarrow$ true 

\item in terms of possible worlds: negation maps a set of possible
  worlds to its complement set
 
\item looking at the behaviour of negation in dialogue shows that this
  is only part of the story

\item we will suggest that the interpretation of negation involves
  being able to distinguish positive and negative propositions
\end{itemize} 


}

\frame{

\frametitle{\textit{No} in dialogue}

\textit{Child approaches socket with nail}\\
\begin{tabular}{ll}
Parent: &  No. [``Don't put the nail in the socket.''] \\
& Do(\#n't) you want to be electrocuted? \\
Child: & No. [``I don't want to be electrocuted.'']\\
Parent: & No. [``You don't want to be electrocuted.'']
\end{tabular}

}  



%\section{Negative questions and answers}
\frame{

\frametitle{Negative questions}
\bit
\item 
Classically the content of $p?$ is identical to
that of $\neg p ?$
\citep{hamblin73,gs97}.

\item Derives from view that the contents of questions are the sets of
  propositions corresponding to their answers

\item Our view is that while the sets of propositions corresponding to
  the \textit{answers} to positive and negative
  questions are the same, the \textit{contents} of the questions are
  distinct.

\end{itemize}

}

\frame{

\frametitle{Examples of embedded questions}

\begin{itemize} 

\item The child wonders whether 2 is even. \\
The child wonders whether 2 isn't  even. \\
(There is evidence that 2 is even) \hfill
\cite{hoepelmann83}
\pause \item  Epstein is investigating whether Strauss-Kahn should be
exonerated \\
Epstein is investigating whether Strauss-Kahn shouldn't be exonerated \\
(There is evidence that Strauss-Kahn should be exonerated.) \hfill
\cite{CooperGinzburg2011,CooperGinzburg2011a}

\pause \item Suggestion that there is reason to believe the positive

\item This holds for negation in assertions as well

\end{itemize}
}

\frame{

\frametitle{Fillmore's frames and resources}

\begin{itemize} 
 
\item Her father doesn't have any teeth 
 
\pause 
\item \# Her husband doesn't have any walnut shells

\pause 
\item Your drawing of the teacher has no nose/\#noses

\pause 
\item The statue's left foot has no \#toe/toes
 
\end{itemize} 

\cite{Fillmore1985}

} 

\frame{

\frametitle{Creating an expectation within a dialogue}

\textit{Resources local to a dialogue}

\bigskip

A: My husband keeps walnut shells in the bedroom.\\
B: Millie's lucky in that respect. Her husband doesn't have any
  walnut shells.

}

\frame{

\frametitle{Distinguishing between positive and negative propositions}

\begin{itemize} 
 
\item The content of \textit{no} is different 
 
\item The raising of a contrary expectation is only with negative
  propositions

\pause \item Some languages have different words for \textit{yes} depending
  on positive and negative propositions
 
\end{itemize} 

}

\frame{

\frametitle{Different words for \textit{yes}}

\begin{itemize} 
 
\item 
\textit{French}\\
A: Marie est une bonne \'etudiante B: Oui / \ugh Si.\\
A: Marie n'est pas une bonne \'etudiante B: \ugh Oui /  Si.


 
\item German \textit{ja/doch}, Swedish \textit{ja/jo}, \ldots 
 
\end{itemize} 

}

\frame{

\frametitle{Possible worlds and the positive/negative distinction}

\begin{itemize} 
 
\item If a proposition is a set of possible worlds, the negation is
  the complement of that set 
 
\pause \item No way of distinguishing ``positive'' and ``negative'' sets of
  possible worlds

\pause \item Perhaps negation is a syntactic rather than a semantic property
(cf recent work by Farkas and Roelofsen)

\pause \item There are lots of ways of making a negative sentence,
\textit{not}, negative quantifiers (\textit{no}, \textit{none},
\textit{nothing}), French \textit{(ne)\ldots pas/point/rien}:
\textit{je n'en sais rien/ j'en sais rien} (``I know nothing about
it''), Swedish words for \textit{not}:  \textit{inte, ej}

\pause \item What makes all these morphemes into morphemes that create
negative sentences?

\pause \item Answer:  the semantic property that they introduce
negative propositions
 
\end{itemize} 

}

    
  



\frame{

\frametitle{Three desiderata for an adequate theory of negation}

\begin{description}

\item[Desideratum 1] $p?$ and $\neg p?$ 
are distinct though their answers correspond to the same set of propositions

\pause \item[Desideratum 2] $\neg p$ implies that there is evidence
that $p$

\pause \item[Desideratum 3] positive and negative propositions can be
distinguished 



\end{description}
}
% \frame{\frametitle{p? $\neq \neg p ?$}

% \begin{itemize}

% \item (\ex{1}a,b)
% seem to describe distinct investigations, the
% first by someone potentially even handed, whereas the second by
% someone tending towards DSK's innocence.
% (cf. original inspiration by )

% \eeen{%
% \item 
%  }
% \end{itemize} 

% }

% \frame{

% \frametitle{Responses to (`Did..?')/ (`Didn't..?') }

% \textit{BNC}

% \medskip

% \begin{tabular}{|p{.17\textwidth}|p{.17\textwidth}|p{.17\textwidth}|p{.17\textwidth}|p{.17\textwidth}|}\hline
% Question type        & Positive answer & Negative answer & No answer &Total\\ \hline \hline
% \texttt{Positive polar}	& 53\% &  31\% & 16\% & n = 106     \\ \hline
% \texttt{Negative polar}	& 23\% &  54\% & 22\% & n = 86  \\ \hline
% \end{tabular}

% \medskip

% \pause almost mirror image distribution


% }

%\section{Negation in simple dialogue}  
% \frame[allowframebreaks]{\frametitle{Today's Talk}
% \begin{itemize} 
 
% \item \cite{gs00}: polar  questions as 0-ary propositional
% abstracts, combined with a theory of
% negative situation types (\cite{cooper98}).
% \item This enables a principled distinction between PPInt and NPInt denotations and presuppositions
% while capturing the identity of resolving answerhood conditions.
% \item 
% Their account relied on a complex, {\it ad hoc} notion of simultaneous
% abstraction.
% \item {\bf Today}:
% \bit
% \item We consider a number of phenomena relating
% negation and dialogue.
% \item We develop an account of
% propositional negation in the framework of Type Theory with Records (TTR)
% \cite{cooper2005,cooper-ddl}. 

% \item This account extends the earlier results
% in a type theoretic framework, based on standard notions of negation
% and abstraction.
% \item First pass account of the
% coherence of propositional negation in dialogue---negation is a
% dialogical notion requiring a rich notion of relevance.
% \eit
% \end{itemize} 
 
% }

% \frame{

% \frametitle{Negation in dialogue}

% \begin{itemize} 
 
% \item \mbox{}
%  [child B approaches socket with nail]\\
% A:  No. 
%  Do(\#n't) you want to be electrocuted? \\
% B: (3) No. \\
% A: (4) No.  
 
 
% \pause 
% \item
% A:  Did Merkel threaten Papandreou?\\
% B:  No. \\
% A:  That can't be true. \\
% B/C: No.


% \pause 
% \item 
% A: Marie est une bonne \'etudiante B: Oui / \ugh Si.\\
% A: Marie n'est pas une bonne \'etudiante B: \ugh Oui /  Si.

% \end{itemize}

% }



% \frame{

% \frametitle{

% \begin{itemize} 
 
% \item {\bf Positive/negative polar question distinction}: \label{desideratum:posnegq}\\
% \item {\bf Constructive negation and negative situation
%   types}: $s : \neg T$  implies $s : T'$, with T' a positive type that
% precludes T \label{desideratum:precl}\\
% \pause 
% %viz. asks for evidence for the existence of a positive situation type alternative
% %to the situation type of $p$\\
% \pause \item {\bf Negative propositions}:  \label{desideratum:NegProp}\\
% \pause \item {\bf Equivalence but non-identity of $p$ and $\neg \neg p$}.\label{desideratum:non-id}

% \end{itemize} 

% }    

% \frame{

% \begin{itemize} 
 
% \item {\bf Constructive negation and negative situation
%   types}: $s : \neg T$  implies $s : T'$, with T' a positive type that
% precludes T \label{desideratum:precl}\\

% \item Additional motivation for
% this is provided by complements of naked infinitive clauses discussed
% below,
% \item  and  the large body of work on the processing of negation, reviewed
% recently in \cite{kaup2006rev} and refined recently in \cite{tian10},
% offering experimental evidence that
% comprehending a negative sentence (e.g.\ \textit{Sam is not wearing a hat})
% involves simulating a scene consistent with the negated sentence.

% \end{itemize} 

% }    

\section{Negation of types}

\frame{

\frametitle{Negation of types}

\begin{itemize} 
 
\item if $T$ is a type then $\neg T$ is a type
 
\pause 
\item distinguishes negative types from positive ones

\pause 
\item $\mathit{cl}_{\neg}(\textit{RecType})$ -- type of the closure of
  record types under negation

\pause 
\item $\mathit{map}_{\neg}(\textit{RecType})$ -- type of singly
  negated record types

\pause 
\item
$\mathit{cl}_{\neg}(\mathit{map}_{\neg}(\textit{RecType}))$ -- type of
negated record types
 
\end{itemize} 
  
}

\frame{

\frametitle{Witnesses for negative types}

\begin{itemize} 
 
\item $a:\neg T$ iff there is some $T'$ such that $a:T'$  and $T'$
  precludes $T$
 
\pause \item $T'$ \textit{precludes} $T$   iff either

\begin{enumerate} 
 
\item $T=\neg T'$
 
\item or, $T,T'$ are non-negative and there is no $a$ such that $a:T$ and $a:T'$ for any
  models assigning witnesses to basic types and ptypes 
 
\end{enumerate}

\pause \item \textit{Remark 1:}  $a:\neg\neg T$ iff $a:T$

\pause \item \textit{Remark 2:} $a:T\vee\neg T$ is \textit{not} necessary --
  $a$ may not be of type $T$ and there may not be any type which
  precludes $T$ either.

\pause \item hybrid classical and intuitionistic negation
   
 
\end{itemize} 
  
}

\frame{

\frametitle{Witnesses for negative types refined}

\begin{itemize} 
 
\item  $a:\neg T$ iff there is some $T'$ such that $a:T'$  and $T'$
  precludes $T$ \textit{and there is some expectation that $a:T$}
 
\item some question of whether this addition should be included here
  or in some theory of when agents are likely to make judgements 
 
\end{itemize} 

}

\frame{

\frametitle{Expectations}

\begin{itemize}

\item What does it mean for there to ``be some expectation''?  

\item Recall the kind of
functions we used to predict completions of events, grammar rules:
dependent types.

\item Discussion relating these dependent types to Aristotelian enthymemes
in 
  \citep{Breitholtz2010,BreitholtzCooper2011}

\item Set of such dependent types are part of general or local resources. 
  
\end{itemize}

}  

\frame{

\frametitle{Austinian propositions}

\begin{itemize} 
 
\item 
\record{\tfield{sit}{\textit{Rec}} \\
              \tfield{sit-type}{$\mathit{cl}_{\neg}$(\textit{RecType})}} -- type of Austinian propositions 
\pause 
\item \record{\field{sit}{$s$} \\
                      \field{sit-type}{\smallrecord{\tfield{c$_{\mathrm{run}}$}{run(sam)}}}} -- an Austinian proposition
                    


\pause 
\item \record{\tfield{sit}{\textit{Rec}} \\
              \tfield{sit-type}{$\mathit{cl}_{\neg}$($\mathit{map}_{\neg}$(\textit{RecType}))}}
            -- type of negative Austinian propositions,
            \textit{NegProp}

\pause \item \record{\tfield{sit}{\textit{Rec}} \\
              \tfield{sit-type}{\textit{RecType}}}  -- type of
            positive Austinian propositions, \textit{PosProp}


 
\end{itemize} 

}




% \frame{

% \frametitle{Austinian witness}

% \begin{itemize} 
 
% \item If $T$ is a record type, then $s$ is an Austinian witness for
%   $T$ iff $s:T$ 
 
% \item If $T$ is a record type, then $s$ is an Austinian witness for
%   $\neg T$ iff $s:T'$ for some $T'$ incompatible with $T$

% \item If $T$ is a type $\neg\neg T'$ then $s$ is an Austinian witness for $T$ iff $s$
% is an Austinian witness for $T'$

% \item The intuitions behind clauses 2--3 are based on an
% intuitive account of witnessing intuitionistic negation.  
% \end{itemize} 

% }

\frame{

\frametitle{Negation of Austinian propositions}

\begin{itemize} 
 
\item \record{\field{sit}{$s$} \\
                      \field{sit-type}{$T$}} 
 
\item \record{\field{sit}{$s$} \\
                      \field{sit-type}{$\neg T$}} 
 
\end{itemize} 

}

\frame{

\frametitle{Truth for Austinian propositions}

An Austinian proposition $p$ is \textit{true} iff
$p.\textrm{sit}:p.\textrm{sit-type}$

}

%\section{Alternatives}

% \frame{

% \frametitle{Perception complements and infonic negation}

% \begin{itemize} 
 
% \item Ralph saw Mary serve Bill 
 
% \pause 
% \item Saw(R,s) $\wedge$ s : Serve(m,b)

% \pause 
% \item Ralph saw Mary not serve Bill

% \item Ralph saw Mary not pay her bill

% \item Saw(R,s) $\wedge$ s : $\neg$ Serve(m,b)

% \item Saw(R,s) $\wedge$ $s \not: Serve(m,b)$
 
% \end{itemize} 

% }

% \frame{

% \frametitle{Alternative positives for infonic negation}

% \cite{cooper98}

% \begin{itemize} 
 
% \item $\forall s,\sigma [s : \overline{\sigma} \ \hbox{\rm implies}$
%   \\ \hspace*{2em} $\exists 
% (\mathit{Pos})\psi[ s : \psi \ \ \hbox{\rm and} \ \  \psi \Rightarrow 
% \overline{\sigma}]]$
% \item $\forall s,\sigma [s : \overline{\sigma} \ \hbox{\rm implies}$
%   \\
% \hspace*{2em} $\exists 
% (\mathit{Pos})\psi[ s : \psi \ \ \hbox{\rm and} \ \ \psi >  \sigma]]$

 
% \end{itemize} 


% }

% \frame{

% \frametitle{Alternative positives in terms of Austinian
%   witnesses}

% \textit{Revise definition of Austinian witness:}

% \medskip

% If $T$ is a record type, then $s$ is an Austinian witness for
%   $\neg T$ iff $s:T'$ for some $T'$ incompatible with $T$ \textit{and there is
%   some $T''$ such that $s:T''$ and $s:T''$ creates an expectation that
%   $T$ is non-empty}

% \bigskip

% \begin{itemize}

% \item What does it mean to ``create an expectation''?  

% \item Recall the kind of
% functions we used to predict completions of events, grammar rules:
% dependent types.

% \item Discussion relating these dependent types to Aristotelian enthymemes
% in 
%   \citep{Breitholtz2010,BreitholtzCooper2011}

% \item Set of such dependent types are part of general or local resources. 
  
% \end{itemize}
% }

   
        


% \section{Positive and negative questions, negative answers}

% \frame[allowframebreaks]{
% \frametitle{Questions as functions returning Austinian propositions}

% \begin{itemize} 
% \item \textit{Do (Don't)  you want to be electrocuted?}

% %\pause 
% \item  $\lambda r$:\textit{Rec} (\ba\[sit = s\\sit-type = \[c
%   : want(B(electrocute(B)))\]\]\ea)

% %\pause 
% \item
%  $\lambda r$:\textit{Rec} (\ba\[sit = s\\sit-type = \[c
%   : $\neg$want(B(electrocute(B)))\]\]\ea)

% %\pause 

% \item \textit{PosPolQ} = \ba \[ p : PosProp\\ 
%                                               q : (Rec)Prop$_{p}$\] \ea

% %\textit{Rec}$\rightarrow$\textit{PosProp}

% \item \textit{NegPolQ} = \ba \[ p : NegProp\\ 
%                                               q : (Rec)Prop$_{p}$\] \ea

% %\textit{Rec}$\rightarrow$\textit{NegProp}

% %\pause 
% \item $T$ is an \textit{atomic answer} for $Q$ iff for some $r$,
%   $Q(r)=T$.  For polar questions there is exactly one simple answer.

% \end{itemize}

% }

% \frame{

% \frametitle{Relating to negative questions}

% \begin{itemize}
% \item  
% Wondering about \ba $\lambda r$:\textit{Rec} (\[sit = s\\sit-type =
% $\neg$T\])\ea  -- wondering about whether (or presupposing that) $s$ has 
% characteristics that typically involve $T$ being the case
% \begin{itemize} 
 
% \item I wonder whether two isn't even 
 
% \item I wonder whether you don't want to electrocute yourself 
 
% \end{itemize} 
  
% \pause \item  The {\it
%   simple answerhood} relation of \cite{gs00} which we saw yesterday ensures
% that the exhaustive answer to p? are $\{p,\neg p\}$, whereas to $\neg
% p?$ they are $\{\neg p, \neg \neg p\}$, so the exhaustive answers are
% equivalent.

 
% \end{itemize} 

% }

% % \frame{

% % \bit
% % \item Assuming that witnessing $\neg T$
% % involves the existence of $T''$ such that $s:T''$ and $T''>T$. \ldots
% % \item 
% % \eit
% % }
% % \section{Characterizing contexts for negation}

% \frame{

% \frametitle{Content for \textit{no} in different dialogue contexts}
% \bit
% \item (context: child about to put nail in socket) Parent: No!

% \item \textit{no} in response to a predicted outcome of an observed
%   event: 

% \item cf reasoning about the game of Fetch
% \end{itemize}

% }

% \frame{


%  \begin{avm}
%  \avml\[
%  phon : {\tt no }\\
%  cat.head = {\it interj} : syncat\\
% {\sc arg-st} = \< \ \> : elist(synsem)\\
% dgb-params : \[ spkr : Ind\\
% addr : Ind\\
% o = \[sit = s\\
% irr-sit-type = \\ (r : \[t : Time\]) T \] : Outcome \]\\
% cont= Want(spkr,$\neg$ Fulfill(o) )   \]\avmr
%  \end{avm}

% \medskip

% Or is the content: $\neg$ Want(spkr, Fulfill(o) ) ?
%  }


% \frame[allowframebreaks]{
% \frametitle{Content for \textit{no} in different dialogue contexts}
% \bit

% \item content of \textit{no} is $\neg T$ if MaxQUD : PosQ and $T$ is an
% atomic answer for MaxQUD

%  \item content of \textit{no} is $T$ if MaxQUD : NegQ and $T$ is
% an atomic answer for MaxQUD

% \item EnsureNeg(p,maxqud) $\leftrightarrow$   p =  q([]) : NegProp;
%   otherwise q([]) : PosProp and p = $\neg q([])$

 

% \begin{avm}
%  \avml\[
%  phon : {\tt no }\\
%  cat.head = adv[+ic] : syncat\\
% {\sc arg-st} = \< \ \> : elist(synsem)\\
%  dgb-params.max-qud : PolQuestion\\
%  cont : NegProp \\ 
%  c1: EnsureNeg(cont, maxqud)
%    \]\avmr
%  \end{avm}
 




% \eit
% }

% \ignore{
% \frame{

% \frametitle{Content for \textit{no} in different dialogue contexts}
% \bit
% \item \textit{no} in response to a predicted outcome of an observed
%   event: content is $\neg T_{\mathrm{predicted}}$

% \pause \item content of \textit{no} is $\neg T$ if MaxQUD : PosQ and $T$ is an
% atomic answer for MaxQUD

% \pause \item content of \textit{no} is $T$ if MaxQUD : NegQ and $T$ is
% an atomic answer for MaxQUD

% \pause \item content of \textit{no} is $\neg T$ if content of previous
% utterance is $T$ and $T$ : \textit{PosProp}

% \pause \item content of \textit{no} is $T$ if content of previous
% utterance is $T$ and $T$ : \textit{NegProp}

% \pause \item \textit{Remark}: content of \textit{no} is always of type \textit{NegProp}

% \end{itemize}

% % \begin{avm}
% % \avml\[
% % phon : {\tt no }\\
% % cat.head = adv[+ic] : syncat\\
% % %{\sc arg-st} = \< \ \> : elist(synsem)\\
% % dgb-params.max-qud : PosQ\\
% % cont=$\neg$(dgb-params.m : NegProp \\ 
% % c1 : SimpleAns(cont,max-qud)
% %    \]\avmr
% % \end{avm}
% }



%\frame{

% \frametitle{`No!' and `No.'}
% \bit
% \item NegVol `no' merely presupposes
% an event/situation concerning which the speaker can express her
% disapproval.

% \item Propositional `No'  can be specified as in (\ex{1}): `No' resolves to a negative
% proposition, which is a {\it simple answer} to MaxQUD.

% \eenumsentence{\label{yes124}\item[]
% \begin{avm}
% \avml\[
% phon : {\tt no }\\
% cat.head = adv[+ic] : syncat\\
% %{\sc arg-st} = \< \ \> : elist(synsem)\\
% dgb-params.max-qud : PolQuestion\\
% cont : NegProp \\ 
% c1: (cont = maxqud([])) $\vee$ \\ 
% $ maxqud([]) \not : NegProp \wedge cont = \neg maxqud([])$
%   \]\avmr
% \end{avm}
% }

% \eit
% }

% \frame{

% \frametitle{polar-question QUD--incrementation}

% \bit
% \item Propositional `No.' uses require the QUD--maximality of
% p?, where $p$ is the proposition they affirm/negate.
% \item  In KoS
% \cite[for example]{ginzburg-nlphandbook,ginzburg-buke}, the felicity
% of these particles in a post-assertoric or post-polar query context is
% assured by the following update rule:


% \begin{avm}\[pre :   \[spkr: Ind\\
%                              addr: Ind\\
%                              p : Prop\\
%                            LatestMove.cont = \\
%                            Ask(spkr,addr,p?) \\ $\vee$ Assert(spkr,addr,p) : IllocProp\\
%                                      \] \\
%                   effects : \[  qud =  \<p?, pre.qud \> :
%                   list(Question) \]
%                   \]\end{avm}

% \eit

% }
%}}

\ignore{
\frame{
\frametitle{The coherence of negative sentences}
\begin{itemize} 

\item Negation triggered by a wide range of dialogical
  sources, including domain knowledge based expectations, 
self- and other- repair:
 
\eeen{\item The army will only confirm that missiles have fallen in Israel \ldots
It was not a chemical attack \ldots     [S2B-015\#106]  (Pitts' [137] )

%\item  I haven't got enough hours in the day \ldots unless I start teaching at midnight.  
%But the studio's not open then.    [S1A-083\#170] (Pitts' [141] )
\item  I might have to do the after-dinner speech at our annual, well, not annual,  
our Christmas departmental dinner.   (Pitts' [112] )

\item A:  there's lots of deers and lots of rabbits.  
  B:  It's not deers - it's deer.    [S1A-006\#261] (Pitts' [107] )}

 \item[] Examples from \cite{pitts-diss}, who collected them
from the International Corpus of English (GB)
\end{itemize} 
}



\frame{
\frametitle{Negation and Relevance}
\bit
\item Given a dialogue gameboard dgb0, a negative proposition $\neg
  p$ is felicitous in dgb0 iff the 
  move `A ask q' is relevant in dgb0 where About(p,q) holds.

\item presupposes rich dialogical notion of relevance (see e.g.\
  Ginzburg PozDial 2010, Ginzburg 2012) or question raising (e.g.\
  {\it Inferential Erotetic Logic} (IEL),  \cite{wisniewski-la}).
\eit

}
}
\ignore{
\frame{
\frametitle{Metalinguistic negation example}
\bit
\item In KoS an utterance $u$ by A in which $u1$ is a sub--utterance of $u$
permits B to accommodate in $u$'s immediate aftermath the issue
(\ex{1}a). (See Ginzburg, Fern\'andez, and Schlangen, tomorrow)
\item  This is {\it inter alia} the basis for explaining why
(\ex{1}b) is a coherent follow up to (\ex{1}a) 



\eeen{\item What form did $A$ intend in $u$1?
\item A: There's lots of deers there.
\item B: Deers? (= Did  $A$ intend the form `deers' in $u1$?)
\item  B:  It's not deers - it's deer.}

\item (\ex{0}d) is About (\ex{0}a) and, hence, felicitous.
\eit
}
}


% \frame{

% \frametitle{

% \begin{itemize}

% \item \textit{no} as agreement to negative assertion

% Cont (`No) = \neg maxqud([]) if  maxqud([]) : PosProp

% Cont (`No) =  maxqud([]) if  maxqud([])  : NegProp



% \eit
% }

% \frame[allowframebreaks]{

% \frametitle{polar-question QUD--incrementation}

% \bit
% \item Propositional `No.' uses require the QUD--maximality of
% p?, where $p$ is the proposition they affirm/negate.
% \item  In KoS
% \ignore{\cite[for example]{ginzburg-nlphandbook,ginzburg-buke}}, the felicity
% of these particles in a post-assertoric or post-polar query context is
% assured by the following update rule:

% \newpage

% \begin{avm}\[pre :   \[spkr: Ind\\
%                              addr: Ind\\
%                              p : Prop\\
%                            LatestMove.cont = \\
%                            Ask(spkr,addr,p?) \\ $\vee$ Assert(spkr,addr,p) : IllocProp\\
%                                      \] \\
%                   effects : \[  qud =  \<p?, pre.qud \> :
%                   list(Question) \]
%                   \]\end{avm}

% \eit

% }
% \frame{
% \frametitle{The coherence of negative sentences}

% \bit


% \item What of the VP adverb `not', in other words sentential
% negation?
% \item  The update rule we just saw provides a class of
% contexts in which clauses of the form `NP $\neg$ VP' are felicitous,
% namely ones in which p? is \mq, where p = cont(`NP VP'). 
% \item However, this
% characterization is partial, as demonstrated by examples like
% (\ex{1}), all drawn from (\ignore{\cite{pitts-diss}}), who collected them
% from the International Corpus of English (GB)


% \eit
% }



% \frame{

% \begin{itemize} 
 
% \item The army will only confirm that missiles have fallen in Israel \ldots
% It was not a chemical attack \ldots     [S2B-015\#106]  (Pitts' [137] )

% \item  I haven't got enough hours in the day \ldots unless I start teaching at midnight.  
% But the studio's not open then.    [S1A-083\#170] (Pitts' [141] )

% \item A:  there's lots of deers and lots of rabbits.  
%   B:  It's not deers - it's deer.    [S1A-006\#261] (Pitts' [107] )
% \item  I might have to do the after-dinner speech at our annual, well, not annual,  
% our Christmas departmental dinner.   (Pitts' [112] )

 
% \end{itemize} 


% }

% \frame{

% \begin{itemize}


% \item Given a dialogue gameboard dgb0, a negative proposition $\neg
%   p$ is felicitous in dgb0 iff the 
%   move `A ask q' is relevant in dgb0 where About(p,q) holds.

% \item presupposes rich dialogical notion of relevance (see e.g.\
%   Ginzburg PozDial 2010, Ginzburg 2012) or question raising (e.g.\
%   {\it Inferential Erotetic Logic} (IEL),  \cite{wisniewski-la}).

% \end{itemize}

% }

% \frame{
% \bit
% \item A key component of the analysis in IEL is
% the use of {\it m(ultiple)-c(onclusion) entailment}---the truth of a set X of premises
% guarantees the truth of at least one conclusion. 
% \item  {\it
% question evocation} can be defined as in (\ex{1}):

% \eeen{\item[] $X$ evokes a question $Q$ iff $X$ mc-entails $dQ$, the set
%   of atomic answers of $Q$, but for no $A\in dQ, X
%   \models A$
% }
% \eit
% }
% \frame{
% \bit
% \item  According to this definition (\ex{1}a) evokes (\ex{1}b) and (\ex{1}c)
%  is About (\ex{1}b). 

% \eeen{\item Missiles have fallen in Israel.
% \item What kind of missiles were fired?
% \item It was not a chemical attack.
% }
% \eit
% }
% \frame{
% \bit
% \item In KoS an utterance $u$ by A in which $u1$ is a sub--utterance of $u$
% permits B to accommodate in $u$'s immediate aftermath the issue
% (a). (See Ginzburg, Fern\'andez, and Schlangen, tomorrow)
% \item  This is {\it inter alia} the basis for explaining why
% (c) is a coherent follow up to (b) and can get the
% resolution (d). 
% \item (e) is About (a)

% \item What form did $A$ intend in $u$1?
% \item A: There's lots of deers there.
% \item B: Deers?
% \item Did  $A$ intend the form `deers' in $u1$?
% \item  B:  It's not deers - it's deer.
% \eit
% }

\frame{
 
\frametitle{Conclusions}

\begin{itemize} 
 
\item Negation in natural language is not a simple ``truth-value flipping''
  truth functional connective
 
\item There are types \textit{Negative Proposition} and
  \textit{Positive Proposition}

\item Positive and negative questions are distinct

\item Negations require positive expectations

\item Exhaustive answers to positive and negative questions are equivalent

\item Distinguishing positive and negative propositions allows a
  straightforward characterization of the content of
  \textit{no}-answers and distinct words for ``yes'' (\textit{oui/si})
  in many languages

%\item The felicity of negative sentences presupposes a rich dialogical
 % notion of relevance

\end{itemize} 


}  




\frame[allowframebreaks]{

\frametitle{References}

\bibliographystyle{apalike}
\bibliography{ams11,/Users/cooper/LaTeX/bib/bibliography}

}

\end{document}




