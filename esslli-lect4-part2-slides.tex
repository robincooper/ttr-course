\documentclass{beamer}

\usepackage{qtree}

\usepackage{beamerthemetree}
\usepackage{color}
\setbeamertemplate{footline}[frame number]
\usepackage{graphicx}
%\usepackage[swedish]{babel} 
\usepackage[latin1]{inputenc} 
\usepackage{natbib}
 
\renewcommand{\newblock}{}
\renewcommand{\bibsection}{}

\title{Generalized quantifiers and clarification}
\author{Robin Cooper  \\
University of Gothenburg \\ \medskip 
Jonathan
Ginzburg \\ Univ. Paris--Diderot, Sorbonne Paris Cit�} 
\date{An Introduction to Semantics using Type Theory with Records \\ ESSLLI
2012 \\ Lecture 4,  part 2}

\AtBeginSection[]
{
   \begin{frame}[plain]
       \frametitle{Outline}
       \tableofcontents[currentsection]
   \end{frame}
}

\newcommand{\backgroundyellow}{\beamertemplateshadingbackground{yellow!40}{magenta!20}}
\newcommand{\backgroundwhite}{\beamertemplateshadingbackground{white!100}{white!100}}

\newcommand{\ignore}[1]{}

%Records

 \newcommand{\record}[1]{$\left[\mbox{\begin{tabular}{lcl} #1
 \end{tabular}}\right]$} 
%\newcommand{\record}[1]{[#1]}

\newcommand{\smallrecord}[1]{$\left[\mbox{\begin{tabular}{@{}l@{}c@{}l@{}} #1
\end{tabular}}\right]$}
\newcommand{\field}[2]{#1 & = & #2}
%\newcommand{\field}[2]{#1=#2}
\newcommand{\tfield}[2]{#1 & : & #2}
\newcommand{\smalltfield}[2]{#1:#2 & &}
%\newcommand{\tfield}[2]{#1:#2}
\newcommand{\mfield}[3]{#1=#2 & : & #3}
\newcommand{\smallmfield}[3]{#1=#2:#3 & &}
\newcommand{\hfield}[2]{{\sc #1} & & #2}

%Types

\newcommand{\down}[1]{[\ \!\check{}\ \!#1]}
\newcommand{\downP}[1]{[\downarrow\!#1]}
\newcommand{\downPl}[2]{[\downarrow_{#2}\!#1]}

\begin{document}

\frame[plain]{\titlepage}

\frame{

\frametitle{Some references}

\cite{Cooper2010a} (paper at SemDial in Pozna\'{n})

\medskip

draft paper submitted to \textit{Dialogue and
  Discourse}
(\url{https://sites.google.com/site/semttr/lectures/gqrch.pdf})

}


\frame[plain]{\frametitle{Outline}\tableofcontents}




%\section{Generalized quantifiers and clarification content}

\section{The Reprise Content Hypothesis}

\frame{

\frametitle{Reprise questions as clarification requests}

\begin{tabular}[t]{lp{.5\textwidth}}
Unknown: & What are you making? \\
Anon 1: & Erm, it's a do- it's \textcolor{blue}{a log}. \\
Unknown: & \textcolor{red}{A log?} 
\end{tabular}

}

\frame{

\frametitle{The Reprise Content Hypothesis (RCH)}

\cite{PurverGinzburg2004,GinzburgPurver2008}

\bigskip

\begin{description}

\item[RCH (weak)] A fragment reprise question queries a part of the standard semantic content 
of the fragment being reprised.

\item[RCH (strong)] A fragment reprise question queries exactly the standard semantic content of 
the fragment being reprised.

\end{description}


}

\section{The anatomy of generalized quantifiers}

\frame{

\frametitle{A type for a ``quantified proposition''}

\begin{itemize} 
 
\item \record{\tfield{restr}{\textit{Ppty}} \\
        \tfield{scope}{\textit{Ppty}} \\
        \tfield{c$_q$}{$q$(restr,scope)}} 
 
\pause \item every($\lambda
r$:\smallrecord{\smalltfield{x}{\textit{Ind}}}(\smallrecord{\smalltfield{c}{dog($r$.x)}}),
$\lambda r$:\smallrecord{\smalltfield{x}{\textit{Ind}}}(\smallrecord{\smalltfield{c}{run($r$.x)}})) 

\pause \item $q$(restriction,scope)

% \pause \item ``propositions'' as types

% \pause \item \textit{Ppty} -- the type of properties:  \smallrecord{\smalltfield{x}{\textit{Ind}}}$\rightarrow$\textit{RecType}

% \pause \item a property -- $\lambda r$:\smallrecord{\smalltfield{x}{\textit{Ind}}}(\smallrecord{\smalltfield{c}{dog($r$.x)}})
 
\end{itemize} 
} 

\frame{

\frametitle{\textit{A thief broke in here last night}}

\begin{itemize} 
 
\item \textit{thief} -- ``the property of being a thief'' 
 
\item \textit{bihln} -- ``the property corresponding to \textit{broke
    in here last night''}

  \item \record{\mfield{restr}{`\textit{thief}'}{\textit{Ppty}} \\
        \mfield{scope}{`\textit{bihln}'}{\textit{Ppty}} \\
        \tfield{c$_\exists$}{$\exists$(restr,scope)}}
 
\end{itemize} 

}

\frame{

\frametitle{Getting back to classical GQ theory}

\cite{BarwiseCooper1981}

\begin{itemize} 
 
\item The \textit{extension} of type $T$, $\down{T}$, is the set $\{a\mid a:T\}$. 
 
\item The \textit{P-extension} of property $P$, $\downP{P}$, is the set \\
$\{a\mid\exists r[r:$\smallrecord{\smalltfield{x}{\textit{Ind}}}
$\wedge\ r.\mathrm{x}=a\wedge\down{P(r)}\not=\emptyset]\}$.

\item The type $q(P_1,P_2)$ is non-empty iff the relation $q^*$ holds between
$\downP{P_1}$ and $\downP{P_2}$.  (Constraint on models)

\item For example:
\begin{itemize} 
 
\item some($P_1$,$P_2$) is non-emtpy (``true'') just in case $\downP{P_1}\cap\downP{P_2}\not=\emptyset$. 
 
\item every($P_1$,$P_2$) is non-emtpy (``true'') just in case
  $\downP{P_1}\subseteq\downP{P_2}$. 

\item many($P_1$,$P_2$) is non-emtpy (``true'') just in case
  $\mid\downP{P_1}\cap\downP{P_2}\mid\ > n$, where $n$ counts as many.
 
\end{itemize} 
  
 
\end{itemize} 

}

\frame{

\frametitle{Witness sets}

\begin{itemize} 
 
\item a witness for a type is something which is of that type

\item a quantifier $q$ is \textit{conservative} means $q^*$ holds
  between $\downP{P_1}$ and $\downP{P_2}$ just in case $q^*$ holds
  between $\downP{P_1}$ and $\downP{P_1}\cap\downP{P_2}$ (\textit{every person
  runs} iff \textit{every person is a person who runs})

\item all natural language quantifier relations are conservative \citep{PetersWesterstahl2006}
 
\item $a:q(P_1,P_2)$ iff $q^*$ holds between $\downP{P_1}$ and $\downP{P_2}$
and $a=\downP{P_1}\cap\downP{P_2}$ 
 
\end{itemize} 

}  
  
% \frame{

% \frametitle{\textit{A thief}}

% \begin{itemize} 
 
% \item $\lambda P$:\textit{Ppty}\\
% \hspace*{1ex}(\record{\mfield{restr}{`\textit{thief}'}{\textit{Ppty}} \\
%         \mfield{scope}{$P$}{\textit{Ppty}} \\
%         \tfield{c$_\exists$}{$\exists$(restr,scope)}}) 
 
% \pause \item Is this the generalized quantifier?

% \pause \item Or is the lambda calculus being used as a glue language?
% (Blackburn and Bos)

% \pause \item If you have another way to do compositional semantics you
% can abandon the lambda technology but still have the generalized
% quantifier relations
 
% \end{itemize} 

% }

% \frame{

% \frametitle{The Purver-Ginzburg alternative anatomy}

% \begin{itemize} 
 
% \item \textit{most students left} 
 
% \item \smallrecord{\smalltfield{q-params}{\smallrecord{\smalltfield{x}{\{\textit{Ind}\}} \\
%                                   \smalltfield{r}{most(x,student)}}} \\
%         \smalltfield{cont}{left(q-params.x)}}

%     \item \cite{Ginzburg2012}

% \pause  \item x is a set containing most students, a witness for the
% quantifier ``most students'' \citep[in the sense of][]{BarwiseCooper1981}

% \pause \item the predicate `left' holds
% (collectively) for that set (which we may interpret as the predicate
% `left' holding individually of each member of the
% witness set)
 
% \end{itemize} 
       
% }

% \frame{

% \frametitle{The Purver-Ginzburg anatomy as a GQ analysis}

% \begin{itemize} 
 
% \item  \smallrecord{\smalltfield{q-params}{\smallrecord{\smalltfield{x}{\{\textit{Ind}\}} \\
%                                   \smalltfield{r}{most(x,student)}}} \\
%         \smalltfield{cont}{left(q-params.x)}}
 
% \pause \item this anatomy is a generalized quantifier analysis 
 
% \pause \item emphasizes the witness set

% \pause \item uses a witness quantifier relation rather than the basic
% quantifier relation

% \pause \item works well for monotone increasing
% quantifiers
% \end{itemize} 

% }

% \frame{

% \frametitle{The Purver-Ginzburg anatomy and monotone decreasing
%   quantifiers}

% \begin{itemize} 
 
% \item \smallrecord{\smalltfield{q-params}{\smallrecord{\smalltfield{x}{\{\textit{Ind}\}} \\
%                                   \smalltfield{r}{few(x,student)}}} \\
%         \smalltfield{cont}{left(q-params.x)}}
  
 
% \pause \item problem pointed out by Purver and Ginzburg

% \pause \item their preferred solution involves treating monotone
% decreasing quantifiers as negations of monotone increasing ones
 
% \end{itemize} 
% }

\frame{

\frametitle{A modification of Ginzburg and Purver}

\begin{description}

\item[type of (potential) witness sets] $a:q^\dagger(P)$ iff $a\subseteq\downP{P}$ and there is some set $X$
such that $q^*$ holds between $\downP{P}$ and $X$.

\item[modified Ginzburg and Purver] \record{\tfield{q-params}{\smallrecord{\tfield{w}{most$^\dagger$(student)}}} \\
        \tfield{cont}{\smallrecord{\mfield{c$_q$}{q-params.w}{most(student,left)}}}} 

\end{description}

}

\frame{

\frametitle{Putting the two analyses together}

\begin{description} 
 
\item[referential] \hspace*{-2em}\smallrecord{\smalltfield{q-params}{\smallrecord{\mfield{restr$_i$}{student}{\textit{Ppty}}
      \\
                                                 \tfield{w$_i$}{most$^\dagger$(q-params.restr$_i$)}}} \\
        \mfield{cont}{ \\
\hspace*{.5em}$\lambda P$:\textit{Ppty} \\
\hspace*{1.5em}(\smallrecord{
                              \mfield{scope}{$P$}{\textit{Ppty}} \\
                              \mfield{c$_{\mathrm{most}}$}{$\Uparrow$q-params.w$_i$}{most($\Uparrow$q-params.restr$_i$,\\
                               & & \hspace*{2.5em}scope)}})}{\textit{Quant}}}
 
\item[non-referential]  \smallrecord{\smalltfield{q-params}{\textit{Rec}} \\
        \mfield{cont}{ \\
\hspace*{.5em}$\lambda P$:\textit{Ppty} \\
\hspace*{1.5em}(\smallrecord{\smallmfield{restr$_i$}{student}{\textit{Ppty}}
      \\
                                                 \smalltfield{w$_i$}{most$^\dagger$(restr$_i$)} \\
                              \smallmfield{scope}{$P$}{\textit{Ppty}} \\
                              \smallmfield{c$_{\mathrm{most}}$}{w$_i$}{most(restr$_i$,scope)}})}{\textit{Quant}}} 
 
\end{description} 

}

% \frame{

% \frametitle{Referential vs non-referential readings}

% \begin{itemize} 
 
% \item Most students will pass the exam \pause \ldots namely the ones
%   who have done all the homework 
 
% \item Sam believes that most students will pass \pause \ldots namely
%   the ones who have done all the homework

% \item Sam doesn't believe that most students will pass \pause \ldots
%   \# namely the ones who have done all the homework
 
% \end{itemize} 

% }    


% \frame{

% \frametitle{\textit{few} as \textit{not many}}


% \begin{itemize} 
 
% \item \smallrecord{\smalltfield{c}{$\neg$(\smallrecord{\smalltfield{q-params}{\smallrecord{\smalltfield{x}{\{\textit{Ind}\}} \\
%                                   \smalltfield{r}{many(x,student)}}} \\
%         \smalltfield{cont}{left(q-params.x)}})}} 
 
% \item my reconstruction of Purver and Ginzburg's suggestion 

% \pause \item what should the content of the noun-phrase \textit{few
%   students} be?

% \pause \item \hspace*{-1.5ex}$\lambda P$:\textit{Ppty}\\ 
% \hspace*{-1ex}(\smallrecord{\smalltfield{c}{$\neg$(\smallrecord{\smalltfield{q-params}{\smallrecord{\smalltfield{x}{\{\textit{Ind}\}} \\
%                                   \smalltfield{r}{many(x,student)}}} \\
%         \smalltfield{cont}{$P$(q-params.x)}})}}) 

% \pause \item if you have a different glue technology you can perhaps
% avoid the lambda abstraction
 
% \end{itemize} 

% }

% \frame{

% \frametitle{Non-monotone quantifiers}

% \begin{itemize} 
 
% \item \textit{only Sam} 
 
% \item \textit{an even number of students}

% \pause \item not clear that the negation trick is available
 
% \end{itemize} 

% }  

% \frame{

% \frametitle{Separating glue from generalized quantifiers}

% \begin{itemize} 
 
% \item Purver and Ginzburg's quarrel is with the lambda technology, not
%   the generalized quantifiers as such 
 
% \item So why not return to the unproblematic quantifier relation and
%   test out the predictions of the first anatomy for clarification? 
 
% \end{itemize} 


% } 

      
% \section{Potential clarification requests and clarifications}

% \frame{

% \frametitle{Possible responses to noun-phrase reprise clarification
%   requests}

% \begin{tabular}[t]{ll}
% A: & A thief broke in here last night \\
% B: & A thief? \\
% A: & \begin{tabular}[t]{lp{.75\textwidth}} \initalphcounter
%       \textit{\alphnum.} & my ex-husband, actually (\textit{witness}) \\
%       \textit{\alphnum.} & burglar wearing a mask (\textit{restriction}) \\
%       \textit{\alphnum.} & got in through the bedroom window (\textit{scope}) \\
%       \textit{\alphnum.} & two, actually (\textit{quantifier relation})
%       \end{tabular}
% \end{tabular}

% }

% \frame{

% \frametitle{Possible non-reprise clarification requests}

% \begin{tabular}[t]{lp{.75\textwidth}}
% A: & Somebody broke in here last night \\
% B: & \begin{tabular}[t]{lp{.75\textwidth}} \initalphcounter
%       \textit{\alphnum.} & (not) your ex-husband? (\textit{witness}) \\
%       \textit{\alphnum.} & burglar wearing a mask? (\textit{restriction}) \\
%       \textit{\alphnum.} & got in through the bedroom window? (\textit{scope}) \\
%       \textit{\alphnum.} & just one? (\textit{quantifier relation})
%       \end{tabular}
% \end{tabular}
% }

% \frame{

% \frametitle{Some odd clarification requests}

% \begin{tabular}[t]{lp{.75\textwidth}}
% A: & Somebody broke in here last night \\
% B: & \begin{tabular}[t]{lp{.75\textwidth}} \initalphcounter
%       \textit{\alphnum.} & maroon? \alphlabel{eg:maroon}\\
%       \textit{\alphnum.} & maroon sweater? \alphlabel{eg:maroonsweater}\\
%       \textit{\alphnum.} & police?  \alphlabel{eg:police}\\
%       \textit{\alphnum.} & scar over the left eye? \alphlabel{eg:scar}\\
%      \end{tabular}
% \end{tabular}

% }

% \frame{

% \frametitle{Clarification requests for ``non-referential''
%   quantifiers}

% \begin{tabular}[t]{lp{.75\textwidth}}
% A: & most thieves are
% opportunists \\
%    &\footnotesize{\url{http://www.accessmylibrary.com/coms2/summary_0286-33299010_ITM},
%   accessed 18th January, 2010}
% \\
% B: & most thieves? \\
% A: & \begin{tabular}[t]{lp{.75\textwidth}} \initalphcounter
%       \textit{\alphnum.} & successful ones
%       (\textit{witness}/\textit{restriction})
%       \alphlabel{eg:successfulthieves}\\
%                          & ``successful
% thieves are opportunists'' (a witness reading) \\
%                         & ``most successful
% thieves are opportunists'' (a restriction reading) \\
%       \textit{\alphnum.} & bide their time (\textit{scope}) \\
%       \textit{\alphnum.} & 80\%, actually (\textit{quantifier relation})
%       \end{tabular} 
% \end{tabular}

% }

% \frame{

% \frametitle{Scope clarifications}

% \begin{itemize} 
 
% \item less likely as clarifications of noun-phrase 
 
% \pause \item  scope note present in the noun-phrase
 
% \end{itemize} 

% }

\section{Clarifications in the Purver-Ginzburg data}

\frame{

\frametitle{What can clarifications address?}

\begin{itemize} 
 
\item must be a path in the record type 
 
\item tends to be a ``major constituent'' 
 
\end{itemize} 


}

\frame{

\frametitle{Predictions for NP clarification requests}

\begin{itemize} 
 
\item witness (for a referential reading) 
 
\item restriction (dispreferred for syntactic reasons)

\item content (possible with restriction or quantifier relation focus)
 
\end{itemize} 


}    



% \frame{

% \frametitle{Kinds of clarification exemplified}

% \begin{itemize} 

% \item clarifications as opposed to clarification requests
 
% \pause \item witness clarifications 
 
% \item restriction clarifications (the majority)

% \item quantifier relation clarifications

% \pause \item no scope clarifications
 
% \end{itemize} 

% }

\frame{

\frametitle{Witness clarifications}

\begin{tabular}[t]{lp{.75\textwidth}}
Unknown: & And er they X-rayed me, and took a urine sample, 
took a blood sample. Er, the doctor  \\
Unknown: & Chorlton? \\
Unknown: & \textbf{Chorlton}, mhm, he examined me, erm, he, he said 
now they were on about a slide $\langle$unclear$\rangle$ on my 
heart. Mhm, he couldn't find it. 
\end{tabular}

\medskip

BNC file KPY, sentences 1005--1008

} 

\frame{

\frametitle{Witness clarifications, \textit{contd.}}

\begin{tabular}[t]{lp{.75\textwidth}}
Terry: & Richard hit the ball on the car. \\
& \ldots \\
Nick: & What ball? \\
Terry: & \textbf{James [last name]'s football}.
\end{tabular}

\medskip

BNC file KR2, sentences 862, 865--866

}

% \frame{

% \frametitle{The status of witness clarifications}

% \begin{itemize} 
 
% \item  \textit{James [last name]'s football} could be a combination of
% \begin{itemize} 
 
% \item a restriction clarification -- \textit{ball}$\rightarrow$\textit{football} 
 
% \item a quantifier relation clarification if we analyze \textit{James [last
%   name]'s} as a determiner representing a quantifier relation 
 
% \end{itemize} 
  
 
% \item \textit{Chorlton} could also be a restriction clarification if
%   you think of the proper name as ``the person named Chorlton'' 
 
% \end{itemize} 

% }

\frame{

\frametitle{Restriction clarifications -- additional material
  (locative relative clause)}

\begin{tabular}[t]{lp{.75\textwidth}}
George: & You want to tell them, bring the tourist around show 
them the spot \\
Sam: & The spot? \\
George: & \textbf{where you spilled your blood}
\end{tabular}

\medskip

BNC file KDU, sentences 728--730

} 

\frame{

\frametitle{Restriction clarifications -- NP with additional modifier
  (relative clause)}

\begin{tabular}[t]{lp{.75\textwidth}}
Terry: & Richard hit the ball on the car. \\
Nick: & What car? \\
Terry: & \textbf{The car that was going past}. \\
\end{tabular}

\medskip

BNC file KR2, sentences 862--864

}

\frame{

\frametitle{Restriction clarifications -- NP with additional modifier
  (noun compound)}

\begin{tabular}[t]{lp{.75\textwidth}}
Anon 1: & In those days how many people were actually 
involved on the estate? \\
Tommy: & Well there was a lot of people involved on the estate 
because they had to repair paths. They had to keep 
the river streams all flowing and if there was any 
deluge of rain and stones they would have to keep 
all the pools in good order and they would \\
Anon 1: & The pools? \\
Tommy: & Yes the pools. That's \textbf{the salmon pools} \\
Anon 1: & Mm.
\end{tabular} 

\medskip

BNC file K7D, sentences 307--313 

}

\frame{

\frametitle{Restriction clarifications -- NP with complement}

\begin{tabular}[t]{lp{.75\textwidth}}
Eddie: & I'm used to sa-, I'm used to being told that at school. 
I want you $\langle$pause$\rangle$ to write the names of these notes 
up here. \\
Anon 1: & The names? \\
Eddie: & \textbf{The names of them}. \\
Anon 1: & Right. 
\end{tabular}

\medskip

BNC file KPB, sentences 417--421

}

\frame{

\frametitle{Restriction clarifications -- NP with noun compound}

\begin{tabular}[t]{lp{.75\textwidth}}
Nicola: & We're just going to Beckenham because we have to 
go to a shop there. \\
Oliver: & What shop? \\
Nicola: & \textbf{A clothes shop}. $\langle$pause$\rangle$ and we need to go to the 
bank too.
\end{tabular} 

\medskip

BNC file KDE, sentences 2214--2217

}

\frame{

\frametitle{Restriction clarification request clarification}

\begin{tabular}[t]{lp{.75\textwidth}} 
Anon 1: & Er are you on any sort of medication at all Suzanne? 
Nothing? \\
Suzanne: & No. Nothing at all. \\
Anon 1: & Nothing? \textbf{No er things from the chemists and cough 
mixtures or anything} $\langle$unclear$\rangle$? 
\end{tabular} 

\medskip

BNC file H4T, sentences 43--48

}

\frame{

\frametitle{Restriction clarifications -- non-monotonic modifier
  substitution}

\begin{tabular}[t]{lp{.75\textwidth}}
Elaine: & what frightened you? \\
Unknown: & The bird in my bed. \\
Elaine: & The what? \\
Audrey: & The birdie? \\
Unknown: & \textbf{The bird in the window}.
\end{tabular}

\medskip

BNC file KBC, sentences 1193--1197

}

\frame{

\frametitle{Restriction clarifications -- non-monotonic restriction
  replacement}

\begin{tabular}[t]{lp{.75\textwidth}}
Mum: & What it ever since last August. I've been treating it as 
a wart. \\
Vicky: & A wart? \\
Mum: & \textbf{A corn} and I've been putting corn plasters on it
\end{tabular}

\medskip

BNC file KE3, sentences 4678--4681

}

\frame{

\frametitle{Restriction clarifications -- monotonic modifier
  substitution}

\begin{tabular}[t]{lp{.75\textwidth}}
Stefan: & Everything work which is contemporary it is 
decided \\
Katherine: & Is one man? \\
Stefan: & No it is a woman \\
Katherine: & A woman? \\
Stefan: & \textbf{A director who'll decide}.
\end{tabular}

\medskip

BNC file KCV, sentences 3012--3016

}

\frame{

\frametitle{Restriction clarifications -- searching for the right
  noun}

\begin{tabular}[t]{lp{.75\textwidth}}
Unknown: & What are you making? \\
Anon 1: & Erm, it's a do- it's a log. \\
Unknown: & A log? \\
Anon 1: & \textbf{Yeah a book, log book}.
\end{tabular}

\medskip

}

\frame{

\frametitle{Restriction clarifications -- difficult to classify}

\begin{tabular}[t]{lp{.75\textwidth}}
Richard: & No I'll commute every day \\
Anon 6: & Every day? \\
Richard: & \textbf{as if, er Saturday and Sunday} \\
Anon 6: & And all holidays? \\
Richard: & Yeah $\langle$pause$\rangle$
\end{tabular}

\medskip

BNC file KSV, sentences 257--261

}

\frame{

\frametitle{Quantifier relation clarifications}

\begin{tabular}[t]{lp{.75\textwidth}}
Anon 2: & Was it nice there? \\
Anon 1: & Oh yes, lovely. \\
Anon 2: & Mm. \\
Anon 1: & It had twenty rooms in it. \\
Anon 2: & \textbf{Twenty rooms?} \\
Anon 1: & \textbf{Yes}. \\
Anon 2: & How many people worked there?
\end{tabular}

\medskip

BNC file K6U, sentences 1493--1499

}

\frame{

\frametitle{Quantifier relation clarifications, \textit{contd.}}

\begin{tabular}[t]{lp{.75\textwidth}}
Marsha: & yeah that's it, this, she's got three rottweilers now and \\
Sarah: & \textbf{three?} \\
Marsha: & \textbf{yeah}, one died so only got three now $\langle$laugh$\rangle$
\end{tabular} 

\medskip

BNC file KP2, sentences 295--297

}  

\section{Conclusions}

\frame{

\frametitle{Generalized quantifiers and the Reprise Content
  Hypothesis}

\begin{itemize} 
 
\item Purver and Ginzburg's data fit the predictions of the classical
  GQ anatomy 
 
\pause \item Unclarities vary within the range of the predictions

\pause \item Responses to clarification questions provide clues to the
  meaning of the clarification request

\pause \item Does this point to the weak version of RCH?

% \pause \item Even stripping away the issue of the lambda glue
% language, when you
% look at the clarifications themselves -- not just the requests, does the Purver-Ginzburg proposal fare any better?
 
\end{itemize} 

}   
\frame[allowframebreaks]

\frametitle{Bibliography}

%\bibliographystyle{acl}

\bibliographystyle{/Users/cooper/LaTeX/bib/mybib}
\bibliography{/Users/cooper/LaTeX/bib/bibliography}

}


  

\end{document}