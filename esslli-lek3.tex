
\documentclass{beamer}

%\usepackage{qtree}
\usepackage{beamerthemetree}
%\usepackage{beamerthemesplit}
\usepackage{color}
\usepackage{lingmacros}
\setbeamertemplate{footline}[frame number]
\usepackage{graphicx}
%\usepackage{xyling}
%\usepackage[swedish]{babel} 
\usepackage[latin1]{inputenc} 
\usepackage{natbib,avm}
\newcommand{\newblock}{}
\usepackage{hyperref}


\title{An Introduction to  Semantics using Type Theory with Records \\
  Lecture 3           }
\author{Jonathan Ginzburg\\
Universit\'e Paris-Diderot, Sorbonne Paris-Cit\'e\\
Robin Cooper\\ 
University of Gothenburg}
\date{}

\AtBeginSection[]
{
   \begin{frame}[plain]
       \frametitle{Outline}
       \tableofcontents[currentsection]
   \end{frame}
}

\newcommand{\backgroundyellow}{\beamertemplateshadingbackground{yellow!40}{magenta!20}}
\newcommand{\backgroundwhite}{\beamertemplateshadingbackground{white!100}{white!100}}
\newcommand{\ba}{\begin{avm}}
\newcommand{\ea}{\end{avm}}
\newcommand{\ugh}{\#}

\newenvironment{display}{\begin{center}}{\end{center}}

%
% Reference the next example in the text:
%
\newcommand{\nexteg}[1]{\addtocounter{examplectr}{1}(\arabic{examplectr}{#1})\addtocounter{examplectr}{-1}}
%
% Reference the previous example in the text:
%
\newcommand{\preveg}[1]{(\arabic{examplectr}{#1})}
%
% Reference examples by relative offsets:
%
\newcommand{\egnum}[2]{\addtocounter{examplectr}{#1}(\arabic{examplectr}{#2})\addtocounter{examplectr}{-#1}}
%
%
% ENVIRONMENTS
%
% Examples Environments
%
\newcounter{examplectr}
\newcounter{subexamplectr}[examplectr]
\renewcommand{\thesubexamplectr}{\alph{subexamplectr}}
%
% Sub-example Macro:
%
\newenvironment{subex}%
   {%\vspace*{-.8\baselineskip}
     \begin{list}
       {\alph{subexamplectr}.}%
       {\setlength{\topsep}{0in}
        \setlength{\leftmargin}{1em}%{0.25in}
        %\setlength{\listparindent}{-2em}
        %\setlength{\labelsep}{\baselineskip}
 \usecounter{subexamplectr} }
       }%
   {\end{list}}
%
% Single Example Macro
%
% \newenvironment{ex}%
%    { \refstepcounter{examplectr}
     
% \bigskip

%     % \begin{list}
%       % {
%        (\arabic{examplectr})%}%
%        % {%\setlength{\topsep}{0in}
% %         \setlength{\leftmargin}{4em}%{0.75in}
% %         \setlength{\labelsep}{\baselineskip}
% % }
%         \hspace*{1em}\begin{minipage}[t]{4in} 
% %\item
%    }%
%    {\end{minipage}

% \bigskip

%     %\end{list}
% }
% %
% % End of example macro
% %
\newcommand{\bit}{\begin{itemize}}
\newcommand{\pa}{\pause}
\newcommand{\eit}{\end{itemize}}
\newcommand{\ben}{\begin{enumerate}}
\newcommand{\een}{\end{enumerate}}
\newcommand{\ignore}[1]{}
\newcommand{\eeen}{\eenumsentence}
\newcommand{\subegref}[2]{(\ref{#1}\ref{#2})}

%\newcommand{\beg}{\begin{ex}}
%\newcommand{\eeg}{\end{ex}}
%\newcommand{\bseg}{\begin{subex}}
%\newcommand{\eseg}{\end{subex}}


\newcommand{\egstack}[1]{\begin{tabular}[t]{@{}l@{}} #1 \\ \\
  \end{tabular}}

%Records

\newcommand{\record}[1]{$\left[\mbox{\begin{tabular}{lcl} #1
\end{tabular}}\right]$} 
\newcommand{\smallrecord}[1]{$\left[\mbox{\begin{tabular}{@{}l@{}c@{}l@{}} #1
\end{tabular}}\right]$}

\newcommand{\field}[2]{#1 & = & #2}
\newcommand{\tfield}[2]{#1 & : & #2}
\newcommand{\smalltfield}[2]{#1:#2 & &}
\newcommand{\mfield}[3]{#1=#2 & : & #3}
\newcommand{\smallmfield}[3]{#1=#2:#3 & &}
\newcommand{\hfield}[2]{{\sc #1} & & #2}




\begin{document}
\maketitle


\ignore{
Lecture 3. A theory of abstract entities and illocutionary interaction: analysis in terms of dialogue game boards; Negation
The semantic ontology is expanded to include abstract entities such as propositions, questions, and outcomes which underpin illocutionary interaction. This provides the background for an analysis of negation that combines aspects of classical, intuitionistic and situation semantics views.


}

\frame{\frametitle{Today's Lecture}

\bit



\item Basic Illocutionary Interaction in KoS
\item Adding abstract entities to the ontology
\item Negation
\eit

}

\section{Basic Illocutionary Interaction in KoS}

\frame{\frametitle{Basic Illocutionary Interaction in KoS}

Based on Chapter 4 from Jonathan Ginzburg. 2011. {\it The Interactive
  Stance: Meaning for conversation}. Oxford University Press.

}

\begin{frame}\frametitle{Dialogue Analysis}

\bit

\item Dialogue analyst's task: describe conventionally acceptable patterns of interaction
(\emph{protocols}), in terms of sequences of information states.

\item Methodological constraint: compositionality (but as with the
sentential level not obsessively [cf.\ the need for
constructionism].

\eit \end{frame}

\frame{\frametitle{Contexts in KoS }

\begin{itemize}


\item in KoS, there is actually no single context, for
reasons that will soon emerge.

\item Instead of a single
context, analysis is formulated at a level of information states, one
per conversational participant. 

\item What the typed entities depicted in
  (\ex{1}),(\ex{2}) amount to is explained  in the next few lectures.

\item The type of information states
is given in (\ex{1}):

\eeen{\item[] TotalInformationState (TIS) =

\ba
\[dialoguegameboard : DGBType\\
private : Private
\]\ea
}


 

\end{itemize}

}
\begin{frame}
\frametitle{Contexts in KoS }

\begin{itemize}

\item The dialogue gameboard (DGB) represents information
that arises from publicized interactions and, for now, we can identify
it as the public context. 

\eeen{\item[]   DGBType (initial definition) =

                 \begin{avm}\[spkr: Ind\\
                             addr: Ind\\
                           c-utt : addressing(spkr,addr)\\
                        Facts : Set(Prop)\\
                           Moves : list(IllocProp)\\
                           QUD : poset(Question)\]\end{avm}

}



\end{itemize}

\end{frame}


\begin{frame}\frametitle{Basics of Interaction}

\bit
\item  The basic units of change
are mappings between DGBs that specify how one DGB configuration can
 be modified into another. $\therefore$ \emph{conversational rule}.

\item  The types specifying its domain and its range respectively the
\emph{preconditions} and  the \emph{effects}.

\eit \end{frame}


\begin{frame}\frametitle{Basics of Interaction}

\bit

\item  Notationwise a conversational rule will be
specified as in (\ex{1}a). We will often notate such a mapping as in
(\ex{1}b):

\eeen{\item r : \begin{avm}\[\ldots\\
                             dgb1 : DGBType\\
                            \ldots\]\end{avm} $\mapsto$
                            \begin{avm}\[\ldots \\
                                         dgb2  : DGBType\\
                                         \ldots\]\end{avm}


\item \begin{avm}\[pre(conds) : RType \\
                  effects : RType \]\end{avm}
}

\eit \end{frame}



\begin{frame}\frametitle{Greeting and Parting}

\bit

\item  The conversational rule associated with greeting:

 \begin{avm}\[pre : \[spkr: Ind\\
                             addr: Ind\\
                             moves = elist : list(IllocProp)\\
                                      qud = elist : list(Question)\\
                                      facts = commonground1 : Prop   \]\\
                  effects :  \[spkr = pre.spkr : Ind\\
                               addr = pre.addr : Ind\\
LatestMove = Greet(spkr,addr):IllocProp\\
                                    qud = pre.qud : list(Question)\\
                                    facts = pre.facts : Prop\]
                  \]\end{avm}





\eit\end{frame}


\begin{frame}\frametitle{Greeting and Parting}

\bit

\item So we can {\it abbreviate}
conversational rules as in (\ex{1}a), which for the rule for greeting
yields (\ex{1}b). 
\eeen{
\item  \ba\[pre: PreCondSpec\\
effects : ChangePrecondSpec\]\ea

\item \begin{avm}\[pre : \[moves = elist : list(IllocProp) \\
 qud = elist : list(Question) \]\\
                  effects :  \[
LatestMove = Greet(spkr,addr):IllocProp\]
                  \]\end{avm}
}

\eit\end{frame}



\begin{frame}\frametitle{Greeting and Parting}


\begin{avm}\[preconds &: \[qud = eset : poset(Question)\\
                            facts :  Set(Prop) \\
                            f  = MinInteraction(\{spkr,addr\}) : Prop\\
                            c1 : member(f,facts) \]\\
                  effects &: \[
LatestMove = Part(spkr,addr): IllocProp\] \]\end{avm}

 \end{frame}

\begin{frame}\frametitle{Greeting and Parting}

\bit
\item Counterparting:

\item \begin{avm}\[preconds &:\[
                             LatestMove =   Part(spkr,addr) : IllocProp\\
                                      qud = eset : poset(Question)
                              \]\\
                  effects &: \[spkr = preconds.addr: Ind\\
                             addr = preconds.spkr: Ind\\
LatestMove =  CounterPart(spkr,addr) : IllocProp
                                  \]
                  \]\end{avm}

\eit
 \end{frame}

\begin{frame}\frametitle{Greeting and Parting}

 \begin{avm}\[preconds &:\[
                             LatestMove =  CounterPart(spkr,addr):IllocProp\\
                                      qud = eset : poset(Question)
                                       \]\\
                  effects &: \[
LatestMove =  Disengaged(\{spkr,addr\}) : IllocProp\\
\]\]\end{avm} 



\end{frame}

\section{Adding abstract entities to the semantic ontology}

\frame{\frametitle{Adding abstract entities to the semantic ontology}

Based on Chapter 3 from Jonathan Ginzburg. 2011. {\it The Interactive
  Stance: Meaning for conversation}. Oxford University Press.

}
\begin{frame}\frametitle{Propositions }

\begin{itemize}


\item Propositional-like entities,
more intensional than events/situations, are a necessary ingredient for accounts of
illocutionary acts, as well as of attitude reports. 
\item  {\it Sets} of situations,
although somewhat more fine grained than sets of worlds, also succumb
to sophisticated variants of logical omniscience (see e.g.\ Soames'
puzzle (\cite{soames86}).). 

\item Building on a conception articulated 30 years earlier by Austin
(\cite{austin50}), \cite{barwise-etchemendy87} developed a theory of
propositions in which a proposition is a structured object
$prop(s,\sigma)$, individuated in terms of a situation $s$ and a
situation type $\sigma$. 

\eeen{\item prop$(s,\sigma)$
is true iff $s : \sigma$ ({\it s is of type} $\sigma$).
\item prop$(s,\sigma)$
is false iff $s \not : \sigma$ ({\it s is not of type} $\sigma$).

}
\end{itemize}

\end{frame}

\begin{frame}\frametitle{Propositions }

\begin{itemize}


\item There are two ways to maintain the insights of an Austinian approach in
 TTR, implicitly Austinian or explicitly so. 

\item \cite{cooper-rlc} develops
 the former in which a proposition $p$ is taken to be a record type. 

\item A
 witness for this type is a situation.

\item On this
 strategy, a witness is not directly included in the semantic
 representation. 

\end{itemize}

\end{frame}

\begin{frame}\frametitle{Propositions }

\begin{itemize}


\item 
record types {\it are} competitive in such a
 role: they are sufficiently fine-grained to
distinguish identity statements that involve distinct constituents. 

\item In this set up substitutivity of co-referentials (\ex{1}e) and
cross-linguistic equivalents ((\ex{1}e), the Hebrew equivalent of
(\ex{1}a)) can be enforced:

\eeen{
\item Enescu is identical with himself.
\item Poulenc is identical with himself.
\item \ba\[ c : Identical(enescu, enescu) \]\ea
\item \ba\[ c : Identical(poulenc, poulenc) \]\ea
\item He  is identical with himself.
\item Enesku zehe leacmo.

}

\end{itemize}

\end{frame}



\begin{frame}\frametitle{Propositions }

\begin{itemize}


\item A situational witness for the record type could also be deduced to explicate cases of
event anaphora, as in(\ex{1}); indeed, a similar strategy will be invoked
when we discuss nominal anaphora. 

\eeen{\label{ev-ana}
%\item A: My back tyre exploded. Two minutes later it started to rain.

\item A: Jo and Mo got married yesterday. It was a wonderful occasion.

\item A: Jo's arriving next week. B: No, that's happening in about a
month. }

\end{itemize}

\end{frame}

\frame[allowframebreaks]{\frametitle{Propositions }

\begin{itemize}


\item Here we develop an explicitly Austinian approach, where
the situational witness is directly included in the semantic
representation. 
\item The original Austinian conception was that $s$ is a
situation deictically indicated by a speaker making an assertion---teasing the semantic difference between implicit and explicit
witnesses is a difficult semantic task.

\item Some other motivation from negation later on today.

\item 
propositions can also play a role in characterizing the communicative
process: in subsequent lectures we will show that ({\it locutionary
  propositions}) individuated in terms of an utterance event $u_0$ as
well as to its grammatical type $T_{u_0}$ allows one to simultaneously
define update and clarification potential for utterances. 

\item In this
case, there are potentially many instances of distinct locutionary
propositions, which need to be differentiated on the basis of the
utterance token---minimally any two utterances classified as being of
the same type by the grammar.

\end{itemize}

}

\begin{frame}\frametitle{Propositions }

\begin{itemize}


\item TTR offers a straightforward way for us to model propositions using
records. A proposition is a  record of the form in (\ex{1}a). The
type of propositions is the record type (\ex{1}b):

\eeen{\label{prop}\item

\ba\[sit =$r_0$ \\ sit-type=$p_0$\]\ea
\item Prop = 
\ba\[sit : Record\\ sit-type : RecType\]\ea}

\pa
\item Truth:

\eeen{\item[] A proposition \ba\[sit =$r_0$ \\ sit-type=$p_0$\]\ea is true iff $r_0 : p_0$ }

\end{itemize} \end{frame}


\begin{frame}\frametitle{  Con/Disjunctive Propositions    }
\bit
\item Given the existence of meet/join operations on types, we can define
conjunction and disjunction operations on propositions, see
\citep{ginzburg-buke} for details 


\eit
\end{frame}

\begin{frame}\frametitle{Questions in TTR}



\begin{itemize}


\item Given the existence of Austinian-like propositions and a theory of
$\lambda$-abstraction given to us by existence of functional types, it is relatively straightforward to develop a
theory of questions as propositional abstracts in TTR, building on
earlier work in situation theory in \cite{ginzburg-lp1,gs00}.

%\item No conflation of questions with properties (=denotations of VPs
  %or common nouns).

\end{itemize} \end{frame}



\begin{frame}\frametitle{Questions: some simple examples}

\begin{itemize}

\item A question will be a function from records into  propositions:

\eeen{\label{q-ex11}

\item who ran

\item TTR representation---(r :  \ba\[x : Ind\\ 
c1 : person(x)\]\ea) \ba\[sit =$r_1$\\
sit-type = \[c : run(r.x)\]\]\end{avm}

 That is, a  function that  maps records r :
T$_{who}$ = \ba\[x : Ind\\ c1 : person(x)\]\ea into
propositions of the form \ba\[sit =$r_1$\\
sit-type = \[c : run(r.x)\]\]\end{avm}
}



\end{itemize} \end{frame}

\begin{frame}\frametitle{Questions: some simple examples}
\eeen{\label{q-ex2}

\item who saw what

\item TTR representation---(r :  \ba\[x : Ind\\ 
c1 : person(x)\\y : Ind \\ c2 : thing(y)\]\ea) \ba\[sit =$r_1$\\
sit-type = \[c : saw(r.x,r.y)\]\]\end{avm}

}
 \end{frame}

\begin{frame}\frametitle{Questions: some simple examples}

\bit
\item And, by extension, a question will be a 0-ary propositional abstract.
\eit
\eeen{\label{q-ex34}
\item Did Bo run

\item TTR representation---( r : \begin{avm}\[ \ \ \]\end{avm})\ba\[sit =$r_1$\\
sit-type = \[c : run(b)\]\]\end{avm}. That is, a
  function that  maps records r :
T$_0$ = \begin{avm}\[ \ \ \]\end{avm}

into propositions of the form \ba\[sit =$r_1$\\
sit-type = \[c : run(b)\]\]\end{avm}
}


\end{frame}


\begin{frame}\frametitle{ Answerhood      }
\bit
\item There are a number of notions of answerhood that are of importance to
dialogue. 

\item We concern ourself here with one that relates to coherence: any
  speaker of a given language can recognize, independently of domain
  knowledge and of the goals underlying an interaction, that certain
  propositions are {\it about} or {\it directly concern} a given
  question.  \eit
\end{frame}



\begin{frame}\frametitle{    Simple Answerhood in TTR   }
 \eenumsentence{

\item[] Given a question $q$ : $(A)B$:

\item AtomAns(q) =$_{def}$ \begin{avm}\[shortans : A \\
                                        propans = q(shortans) : Prop\\
                                        \]\end{avm}

\item NegAtomAns(q) =$_{def}$ \begin{avm}\[shortans : A \\
                                        propans = $\neg$ q(shortans) : Prop\\
                                        \]\end{avm}

\item SimpleAns(q) =$_{def}$ AtomAns(q) $\vee$ NegAtomAns(q)
 }

\end{frame}

\begin{frame}\frametitle{ Simple Answerhood     }
\bit

\item Simple answerhood covers a fair amount of ground.  But it clearly
underdetermines aboutness.  
\item On the polar front, it leaves out the
whole gamut of answers to polar questions that are weaker than $p$ or
$\neg p$ such as conditional answers `If r, then p' (e.g.\ \ex{1}a) or
weakly modalized answers `probably/possibly/maybe/possibly not
p'(e.g.\ \ex{1}b).  
\item As far as wh-questions go, it leaves out
quantificational answers (\ex{1}c-g), as well as disjunctive answers.


\eit
\end{frame}


\begin{frame}\frametitle{ Simple  Answerhood v.\ Aboutness     }
\eeen{\label{corp-ex}

\item Christopher: Can I have some ice-cream then?\\
Dorothy: you can do if there is any. (BNC, KBW)

\item Anon: Are you voting for Tory? \\ Denise: I might. (BNC, KB?,
slightly modified)

\item Dorothy: What did grandma have to catch?\\
Christopher: A bus. (BNC, KBW,
slightly modified)

\item Rhiannon: How much tape have you used up?\\
Chris: About half of one side. (BNC, KB?)

\item Dorothy: What do you want on this?\\
Andrew: I would like some yogurt please. (BNC, KBW,
slightly modified)

\item Elinor: Where are you going to hide it?\\
Tim: Somewhere you can't have it.(BNC, KBW)

\item Christopher: Where is the box?\\
Dorothy: Near the window. (BNC, KBW)
}

\end{frame}


\begin{frame}\frametitle{   Aboutness via disjunction    }
\bit
\item One straightforward way to enrich simple answerhood is to consider the
relation that emerges by closing simple answerhood under
disjunction. See \cite{gs00} for more discussion.



\eeen{\label{dc-simp-ans1}
\item[] $p$ is about $q$ iff $p$ entails a disjunction of simple answers.

}

\eit
\end{frame}


\begin{frame}\frametitle{   Aboutness via disjunction    }
\bit
\item If r : SimpleAns(q), then since r : AtomAns(q) or r :
  NegAtomAns(q), r is a record of the form
  in (i), p1 then is the value r gets on the propans field:

\bit
\item[(i)] r = \ba\[shortans = a\\propans = p
\]\ea
\eit

\eenumsentence{

\item[] Given a question $q$ : $(A)B$:

\item Aboutness(q) =$_{def}$ \begin{avm}\[r1 : SimpleAns(q)\\
p1 = r1.propans : Prop \\
r2 : SimpleAns(q)\\
p2 = r2.propans : Prop \\
                                                  propans  : Prop\\
                                         c : $Entails(propans,
                                         p1 \vee_{prop} p2)$
                                         \]\end{avm}
}
\eit
\end{frame}


\begin{frame}\frametitle{Answering a question with a question}


\eeen{\label{q-ex31}
\item A: Who murdered Smith? B: Who was in town?
\item A: Who is going to win the race? B: Who is going to participate?
\item Carol:        Right, what do you want for your dinner?\\
 Chris:  What do you (pause) suggest? (BNC, KbJ)
 \item Chris:        Where's mummy?\\
 Emma:       Mm? \\
 Chris:     Mummy? \\
 Emma:     What do you want her for? (BNC, KbJ)

}
\bit
\item Detailed discussion of these on Fri in the class on theories of
  questions.
\eit
\end{frame}

\section{Asking, Asserting, Answering, and Accepting}

\begin{frame}\frametitle{Asking, Asserting, Answering, and Accepting}

\bit

\item Broadly speaking queries and assertions are either \emph{issue
initiating}---they introduce an issue unrelated to those currently
under discussion--- or they are \emph{reactive}---they involve a
reaction to a previously raised issue.

\item Accounting for the the reactive ones using
DGB--based conversational rules is simple. These can also be used to
explicate the effects \emph{issue initiating} moves have. More on that a bit later.
\eit\end{frame}


\begin{frame}\frametitle{Simple assertion and querying: ingredients}

{\small
\begin{tabular}{|c|c|}
  \hline
  % after \\: \hline or \cline{col1-col2} \cline{col3-col4} ...
  {\sf  querying  } & {\sf  assertion }\\ \hline
  \hline
  LatestMove =  Ask(A,q) & LatestMove =  Assert(A,p)\\ \hline
A: push q onto QUD; &  A:  push p? onto QUD;\\
release turn; & release turn \\ \hline
B: push  q onto QUD; &  B: push p? onto QUD;\\
 take turn; &  take turn;\\
make q---specific &  Option 1: Discuss p?  \\
utterance & \\
 take turn.& Option 2: Accept p\\ \hline
& LatestMove =  Accept(B,p)\\ \hline

& B: increment FACTS with p;\\
& pop p? from QUD;\\ \hline
& A: increment FACTS with p;\\
& pop p? from QUD;\\ \hline
\end{tabular}
}
\end{frame}

\begin{frame}\frametitle{Simple assertion and querying: ingredients}


\begin{itemize}
\item q-specific utterance: an utterance whose content is either a proposition $p$ {\sf
About} max-qud (\emph{partial answer}) or a question $q_1$ on which
max-qud {\sf Depends} (\emph{sub-question}).





\end{itemize}

 \end{frame}

\begin{frame}[label=qspec]\frametitle{Asking, Asserting, Answering, and Accepting}

\bit

\item Two aspects of this protocol are not query specific:

\ben \item The protocol is like the ones we have seen for greeting
and parting a 2-person turn exchange protocol (2-PTEP).

\item The specification {\tt make
  q-specific utterance} is an instance of a general constraint that characterizes
   the contextual background of reactive queries and
assertions.

\een
\eit
%\hfill\hyperlink{bcm1}{\beamerreturnbutton{Back}}
\end{frame}

\begin{frame}\frametitle{Asking, Asserting, Answering, and Accepting}

\bit
\item QSPEC: if $q$ is
QUD--maximal, then subsequent to this either conversational
participant may make a move constrained to be $q$--specific (i.e.\
either About or Influencing $q$.).



\enumsentence{ {\sf QSPEC}

\begin{avm}\[pre : \[qud = \<q, Q\> : poset(Question)\] \\
                  effects  :  TurnUnderspec $\wedge_{merge}$\\
                   \[
r : AbSemObj\\
R: IllocRel\\
LatestMove = R(spkr,addr,r) : IllocProp\\
                                    c1 : Qspecific(r,q)\\
                                    \]\]\end{avm}
}
\eit







\end{frame}

\begin{frame}\frametitle{Turn Change and abbreviation}

\bit
\item Turnholder-Underspecified = \ba\[ pre : \[ spkr : Ind\\
                                                    addr : Ind\]\\
effects : \[
PrevAud = \{ pre.spkr,pre.addr\}) : Set(Ind)\\
spkr : Ind\\
c1 : member(spkr, PrevAud)\\
addr : Ind\\
c2: member(addr, PrevAud)  $\wedge$
addr $\neq$ spkr\]\]\ea

\eit\end{frame}

\begin{frame}\frametitle{Asking, Asserting, Answering, and Accepting}

\bit


\item The only query specific aspect of the querying protocol is:

\eeen{\item[] Ask QUD--incrementation:

\begin{avm}\[pre : \[   q : Question\\
                             LatestMove = Ask(spkr,addr,q):IllocProp \]\\
                  effects : \[ qud = \<q,pre.qud\> : poset(Question) \]\]\end{avm}
                  }

\eit\end{frame}


\begin{frame}\frametitle{Asking, Asserting, Answering, and Accepting}

\bit




\item What are the components of the assertion protocol? Not specific to
assertion is the fact that it is a 2-PTEP; similarly, the discussion
option is simply an instance of QSPEC.



\item This leaves two novel components: QUD incrementation with $p?$
and acceptance.

\eit\end{frame}

\begin{frame}\frametitle{Asking, Asserting, Answering, and Accepting}


\eeen{\item[] Assert QUD--incrementation:

\begin{avm}\[preconds &:\[  p : Prop\\
                       LatestMove = Assert(spkr,addr,p):IllocProp
                                 \] \\
                  effects &: \[     qud = \<p?,pre.qud\> : poset(Question)
\]\]\end{avm}
                  }
\end{frame}

\ignore{
\begin{frame}\frametitle{  Assertion checking     }
\bit

\item Assertion checking licenses an asking of p? as a follow up of
asserting p, the checker can be either the original asserter or the
original addressee. 
\item An asserter's demand for confirmation does not
introduce a new issue but it does change the context, forcing a
response about the issue from B:


\eeen{\label{check}
\item A: Bo is leaving, right? B: Right / \ugh I see / \ugh Aha.

\item A: Bo is leaving, isn't he?  B: Right / \ugh I see / \ugh Aha.

\item A: Bo is leaving. B: I see / Aha.
}




\eit
\end{frame}
\begin{frame}\frametitle{  Assertion checking     }


\eeen{\item[] Assertion checking:

\begin{avm}\[pre :\[   p : Prop\\
                       LatestMove = Assert(spkr,addr,p):IllocProp\\
                                      qud = \<p?,Q1\> : poset(Question)
                                    \] \\
                  effects :  TurnUnderspec  $\wedge_{merge}$ \\ 
                   \[LatestMove = Check(spkr,addr,p?):IllocProp
                                   \]\]\end{avm}



}


\end{frame}
\begin{frame}\frametitle{Asking, Asserting, Answering, and Accepting}

\bit


\item
Acceptance is a somewhat more involved matter because a lot of the
action is not directly perceptible.

\item The labour can be divided here in two: on the one hand is the
action brought about by an acceptance utterance (e.g.\ `mmh', `I
see').

\item The background for an acceptance by B is an assertion by A and
the effect is to modify LatestMove:

\eit\end{frame}

\begin{frame}\frametitle{Asking, Asserting, Answering, and Accepting}



\eeen{\item[] Accept move:

\begin{avm}\[preconds &:\[  spkr: Ind\\
                             addr: Ind\\
p : Prop\\
                          LatestMove = Assert(spkr,addr,p):IllocProp\\
                                     qud = \<p?,preconds.qud\> : poset(Question)
\] \\
                  effects &: \[
spkr = preconds.addr: Ind\\
addr = preconds.spkr: Ind\\
LatestMove = Accept(spkr,addr,p) : IllocProp\]\]\end{avm}
                  }
\end{frame}
\begin{frame}\frametitle{  Confirming     }
\bit
\item If we distinguish Check moves from Ask moves, we also need
to distinguish  acceptance from confirmation.
\item  Checks cannot be
followed up by acceptance moves, as we saw above. 



\eeen{\item[] {\sf Confirm move}:

\begin{avm}\[pre :\[   p : Prop\\
                          LatestMove = Check(spkr,addr,p?):IllocProp\\
                                     qud = [p?,preconds.qud] : poset(Question)\] \\
                  effects : \[
spkr = preconds.addr: Ind\\
addr = preconds.spkr: Ind\\
LatestMove = Confirm(spkr,addr,p) : IllocProp\]\]\end{avm}
                  }
\eit
\end{frame}
}

\begin{frame}\frametitle{Asking, Asserting, Answering, and Accepting}

\bit


\item
The second component of acceptance is the incrementation of FACTS by
$p$.

\item  This is not quite as straightforward as it might seem: when
FACTS gets incremented, we also need to ensure that p? gets
downdated from QUD---only Nonresolved questions can be in QUD
(see notes for discussion, leads to account of rhetorical questions.). 

\item In order to ensure that this is the case, we need to
check for all existing elements of QUD that they are not resolved by
the new value of FACTS.

\item Hence, accepting p involves both an update of FACTS and a
downdate of QUD---minimally just removing p?, potentially removing
other questions as well.

\eit\end{frame}





\begin{frame}\frametitle{Fact Update/ QUD Downdate}
\bit
\item NonResolve is a
function that maps a partially ordered set of questions $poset(q)$ and a set of propositions $P$
to a  partially ordered set of questions $poset'(q)$ which is identical to $poset(q)$
modulo those questions in $poset(q)$ resolved by members of $P$.

\eeen{\label{0fuqd}
\item[] {\sf Fact Update/ QUD Downdate}


\begin{avm}\[pre :\[p : Prop\\
                             LatestMove = Accept(spkr,addr,p) $\vee$\\ Confirm(spkr,addr,p)\\
                             qud = \<p?,Q\> : poset(Question) \]\\
              effects :\[ facts = pre.facts $\cup$ \{ p \} :  Set(Prop)\\
                          qud =  NonResolve(Q,facts) : poset(Question)  \]  \]\end{avm}
}

\eit

\end{frame}




\begin{frame}\frametitle{What drives the dialogue?}

\bit
\item Baseline rule: {\sf Free Speech}; when QUD is empty one can say
whatever one likes.  
\item This is true when one stands on the proverbial
soap box or (various caveats) when sitting in a restaurant
with friends \ldots
\item \begin{avm}\[ pre :\[
 qud = \< \> : eset(question)
\]\\
effects : TurnUnderspec  $\wedge_{merge}$ \\ \[
r : AbSemObj\\
R: IllocRel\\
LatestMove = R(spkr,addr,r) IllocProp
\]\]\end{avm}
\eit
\end{frame}

\begin{frame}\frametitle{A simple example}

\eenumsentence{\label{ex111}\item A: Hi\\ B: Hi \\ A: Who's coming tomorrow? \\
B:  Several colleagues of mine (are coming).  \\
A:  I see. \\
B:  Mike (is coming) too.  

}
\end{frame}

\begin{frame}\frametitle{A simple example}

{\footnotesize
\begin{tabular}{|c|c|c|}
\hline
Utt. & DGB Update  & Rule \\ \
&  (Conditions) & \\
\hline
initial & MOVES = $\langle  \rangle$ & \\
& QUD = $\langle  \rangle$ & \\
& FACTS = cg1 &\\
1 &  LatestMove := Greet(A,B)  & {\sf greeting} \\
2 &  LatestMove := CounterGreet(B,A) & {\sf countergreeting} \\
3 & LatestMove := Ask(A,B,q0) & {\sf Free
  Speech} \\
& QUD : = $\langle q0 \rangle$ & {\sf Ask  QUD--incrementation} \\
4 & LatestMove := Assert(B,A,p1) & {\sf
QSPEC} \\
& (About(p1,q0)) &\\
& QUD : = $\langle p1?,q0 \rangle$ & {\sf Assert  QUD--incrementation} \\
5 & LatestMove := Accept(A,B,p1) & {\sf Accept} \\
& QUD := $\langle q0\rangle$ & {\sf Fact update/QUD downdate} \\
& FACTS := cg1 $\wedge$ p1 & \\
6 & LatestMove := Assert(B,A,p2) & {\sf
QSPEC} \\
& (About(p2,q0)) & \\
& QUD : = $\langle p2?,q0 \rangle$ & {\sf Assert  QUD--incrementation}\\
\hline
\end{tabular}
}\normalsize

\end{frame}

\begin{frame}\frametitle{ Assertoric Benchmark 2}

\bit
\item \textcolor[rgb]{0.98,0.00,0.00}{Assertion benchmark 2: Accommodate disagreement}


\eeen{\item[] \label{rtsimpq1}       

A(1): Who will agree to come?

B(2): Helen and Jelle. 

A(3): I doubt Helen will want to come after last time.  

B(4): I think she's forgiven and forgotten.

A(5): OK.           
}

\eit

\end{frame}

\begin{frame}\frametitle{ Assertoric Benchmark 2}

{\small \begin{tabular}{|c|c|c|}
\hline
Utt. & DGB Update  & Rule \\ \
&  (Conditions) & \\
\hline
initial & MOVES = $\langle  \rangle$ & \\
& QUD = $\langle  \rangle$ & \\
& FACTS = cg1 &\\
1 & LatestMove := Ask(A,B,q0) & {\sf Free
  Speech} \\
& QUD : = $\langle q0 \rangle$ & {\sf Ask  QUD--incrementation} \\
2 & LatestMove := Assert(B,A,p1) & {\sf
QSPEC} \\
& (About(p1,q0)) &\\
& QUD : = $\langle p1?,q0 \rangle$ & {\sf Assert  QUD--incrementation} \\
3 & LatestMove := Assert(A,B,p2) & {\sf QSPEC} \\
& (About(p2,p1?)) & \\
& QUD := $\langle p2?,p1?,q0\rangle$ & {\sf Assert  QUD--incrementation} \\
4 & LatestMove := Assert(B,A,p3) & {\sf
QSPEC} \\
& (About(p3,p2?)) & \\
& QUD := $\langle p3?, p2?,p1?,q0?\rangle$ & {\sf Assert  QUD--incrementation} \\
5 & LatestMove := Accept(A,B,p3) & {\sf Accept} \\
& QUD := $\langle p1?,q0\rangle$ & {\sf Fact update/QUD downdate} \\
& FACTS := cg1 $\cup$ \{p3\} & \\
\hline
\end{tabular}
 }
\end{frame}

\begin{frame}\frametitle{Query benchmark 4: query responses}

\bit
\item \textcolor[rgb]{0.98,0.00,0.00}{ Query benchmark4: accommodate sub-questions.}




\eeen{\label{qqq} \item[]     

A(1): Who should we invite for tomorrow?

B(2): Who will agree to come?

A(3): Helen and Jelle and Fran and maybe Sunil.

B(4) : (a) I see. (b) So, Jelle I think.

A(5): OK.
}

\end{itemize}\end{frame}

\begin{frame}\frametitle{Query benchmark 4: query responses}

{\scriptsize \begin{tabular}{|c|c|c|}
\hline
Utt. & DGB Update  & Rule \\ \
&  (Conditions) & \\
\hline
initial & MOVES = $\langle  \rangle$ & \\
& QUD = $\langle  \rangle$ & \\
& FACTS = cg1 &\\
1 & LatestMove := Ask(A,B,q0) & {\sf Free Speech} \\
& QUD : = $\langle q0 \rangle$ & {\sf Ask  QUD--incrementation} \\
2 & LatestMove := Ask(B,A,q1) & {\sf
QSPEC} \\
& Influence(q1,q0) &\\
& QUD : = $\langle q1,q0 \rangle$ & {\sf Assert  QUD--incrementation} \\
3 & LatestMove := Assert(A,B,p1) & {\sf QSPEC} \\
& (About(p1,q1)) & \\
& QUD := $\langle p1?,q1,q0\rangle$ & {\sf Assert  QUD--incrementation} \\
4a & LatestMove := Accept(B,A,p1) & {\sf Accept} \\
& FACTS := cg1 $\cup$ \{p1\}\ & \\
& QUD := $\langle q0 \rangle$ & {\sf Fact update/QUD downdate} \\
4b & LatestMove := Assert(B,A,p2) & {\sf QSPEC} \\
& (About(p2,q0)) & \\
& QUD := $\langle p2?,q0\rangle$ & {\sf Assert  QUD--incrementation}\\
5 & LatestMove := Accept(A,B,p2) & {\sf Accept} \\
& FACTS := cg1 $\cup$ \{p1,p2\}\& \\
& QUD := $\langle q0 \rangle$ & {\sf Fact update/QUD downdate} \\
\hline
\end{tabular}
}

\end{frame}

\begin{frame}\frametitle{ Query Benchmark 6 (partially)}


\bit

\item Same basic mechanism seems to regulate queries/assertions, across varying sizes of participant sets:

\eeen{ \item Monologue: self answering (\emph{A: Who should we invite?
    Perhaps Noam.})

 \item Dialogue: querier/responder (\emph{A: Who should we invite?
    B: Perhaps Noam.})

\item Multilogue: discussed later, maybe.
}



\item \textcolor[rgb]{0.98,0.00,0.00}{ Query benchmark 6: ensure
  approach scales down  to monologue  } 

\eit\end{frame}

\nocite{ginzburg-buke}

\bibliographystyle{natbib}
\bibliography{newest-jg-fin}

\end{document}

\ignore{

Lecture 3. A theory of abstract entities and illocutionary interaction: analysis in terms of dialogue game boards; Negation
The semantic ontology is expanded to include abstract entities such as propositions, questions, and outcomes which underpin illocutionary interaction. This provides the background for an analysis of negation that combines aspects of classical, intuitionistic and situation semantics views.


}

