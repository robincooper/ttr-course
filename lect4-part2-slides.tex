\documentclass{beamer}

\usepackage{qtree}

\usepackage{beamerthemetree}
\usepackage{color}
\setbeamertemplate{footline}[frame number]
\usepackage{graphicx}
%\usepackage[swedish]{babel} 
\usepackage[latin1]{inputenc} 
\usepackage{natbib}
 
\renewcommand{\newblock}{}
\renewcommand{\bibsection}{}

\title{Generalized quantifiers and copredication}
\author{Robin Cooper  \\
University of Gothenburg \\ \medskip 
Jonathan
Ginzburg \\ Univ. Paris--Diderot, Sorbonne Paris Cit�} 
\date{Type theory with records for natural language semantics, NASSLLI
2012 \\ Lecture 4,  part 2}

\AtBeginSection[]
{
   \begin{frame}[plain]
       \frametitle{Outline}
       \tableofcontents[currentsection]
   \end{frame}
}

\newcommand{\backgroundyellow}{\beamertemplateshadingbackground{yellow!40}{magenta!20}}
\newcommand{\backgroundwhite}{\beamertemplateshadingbackground{white!100}{white!100}}

\newcommand{\ignore}[1]{}

%Records

 \newcommand{\record}[1]{$\left[\mbox{\begin{tabular}{lcl} #1
 \end{tabular}}\right]$} 
%\newcommand{\record}[1]{[#1]}

\newcommand{\smallrecord}[1]{$\left[\mbox{\begin{tabular}{@{}l@{}c@{}l@{}} #1
\end{tabular}}\right]$}
\newcommand{\field}[2]{#1 & = & #2}
%\newcommand{\field}[2]{#1=#2}
\newcommand{\tfield}[2]{#1 & : & #2}
\newcommand{\smalltfield}[2]{#1:#2 & &}
%\newcommand{\tfield}[2]{#1:#2}
\newcommand{\mfield}[3]{#1=#2 & : & #3}
\newcommand{\smallmfield}[3]{#1=#2:#3 & &}
\newcommand{\hfield}[2]{{\sc #1} & & #2}

%Types

\newcommand{\down}[1]{[\ \!\check{}\ \!#1]}
\newcommand{\downP}[1]{[\downarrow\!#1]}
\newcommand{\downPl}[2]{[\downarrow_{#2}\!#1]}

\begin{document}

\frame[plain]{\titlepage}
\frame[plain]{\frametitle{Outline}\tableofcontents}

\frame{

\frametitle{Reference}

\cite{Cooper2011}

}

\section{Predication, polysemy and co-predication}

\frame{

\frametitle{Predication and apparent polysemy}

\begin{itemize} 
 
\item The lunch was delicious (food)
 
\item The lunch took forever (event)

\item \cite{Pustejovsky1995,AsherPustejovsky2005,Asher2011}

\item John has memorized the score of
    the Ninth Symphony

\item The score of the Ninth Symphony is
    lying on the piano

\item \cite{McCawley1968}
 
\end{itemize} 

}

\frame{

\frametitle{Predication and innovation}

\begin{itemize} 
 
\item The blancmange was delicious 
 
\item $^i$IFK G�teborg's last game was delicious  (coercion of \textit{delicious})

\item $^i$The blancmange took forever  (coercion of \textit{blancmange})

\item IFK's last game took forever
 
\end{itemize} 

}

\frame{

\frametitle{Copredication - an argument against polysemy}

Pustejovsky (and Asher)

\begin{itemize} 
 
\item The lunch was delicious but took forever 
 
\item $^!$The bank specializes in IPO's and is being quickly eroded by
  the river 
 
\end{itemize} 

}

\frame{

\frametitle{\textit{wrong} -- also an argument against polysemy}

\begin{itemize} 
 
\item $^i$Sam went to the wrong bank (coercion of \textit{bank} to be
  the join of two readings, common in jokes, \citealp{Ritchie2004}) 
 
\item Kim told me about the wrong lunch (\#she told me about the
  conversation rather than the food) 
 
\end{itemize} 
    
}

\frame{

\frametitle{The lunch frame type}

\record{\tfield{x}{\textit{Ind}} \\
        \tfield{event}{\textit{Event}} \\
        \tfield{food}{\textit{Food}} \\
        \tfield{c$_{\mathrm{lunch}}$}{lunch\_ev\_fd(x, event, food)}}

}

\frame{

\frametitle{The content of \textit{lunch}}

$\lambda r$:\smallrecord{\smalltfield{x}{\textit{Ind}}}(\smallrecord{
\smalltfield{event}{\textit{Event}} \\
\smalltfield{food}{\textit{Food}} \\
\smalltfield{c$_{\mathrm{lunch}}$}{lunch\_ev\_fd($r$.x, event,
  food)}}) 

}

\section{Dynamic generalized quantifiers}

\frame{

\frametitle{\textit{A donkey runs} -- non-dynamic}

\begin{itemize} 
 
\item \record{\tfield{c$_{\mathrm{exists}}$}{exists($\lambda r$:\smallrecord{\tfield{x}{{\it
          Ind\/}}}(\smallrecord{\tfield{c$_{\mathrm{donkey}}$}{donkey($r$.x)}}),
    \\ & & \hspace*{3em}$\lambda
          r$:\smallrecord{\tfield{x}{{\it
          Ind\/}}}(\smallrecord{\tfield{c$_{\mathrm{run}}$}{run($r$.x)}}))}} 
 
\item  $q(P_1,P_2)$ is non-empty just in case $q*$ holds between
$\downP{P_1}$ and $\downP{P_2}$

\item $q*$ is the classical generalized quantifier relation for $q$

\item if $P$:\textit{Ppty}, $\downP{P}=\{a\mid\exists
  r[r:$\smallrecord{\smalltfield{x}{\textit{Ind}}}$\wedge\
  r.\mathrm{x}=a\wedge\{b\mid b:P(r)\}\not=\emptyset]\}$ (the set of
  objects that have the property, the \textit{property extension})

\item exists($P_1$,$P_2$) is a non-empty type just in case $\downP{P_1}\cap\downP{P_2}\not=\emptyset$  
 
\end{itemize} 


} 

\frame{

\frametitle{Making quantifiers dynamic}

\begin{itemize} 
 
\item We make quantifiers dynamic by passing the information from the
  first property to the second property, thus making the second
  property more restricted 
 
\item instead of exists(``donkey'',``runs'') we have
  exists(``donkey'',``runs restricted to donkeys'')

\item If $f$ is a function $\lambda v:T_1(\phi)$, then the
\textit{restriction of} $f$ \textit{by the type} $T_2$, $f\restriction
T_2$, is $\lambda
v:T_1$\d{$\wedge$}$T_2(\phi)$

\item $q(P_1,P_2)$ is a non-empty type just in case $q*$ holds between
$\downP{P_1}$ and $\downP{P_2\restriction {\cal F}(P_1)}$

\item $\mathcal{F}$($\lambda
  r$:\smallrecord{\smalltfield{x}{\textit{Ind}}}(\smallrecord{\smalltfield{c$_{\mathrm{donkey}}$}{donkey($r$.x)}}))
  = \smallrecord{\smalltfield{x}{\textit{Ind}} \\
                 \smalltfield{c$_{\mathrm{donkey}}$}{donkey(x)}}
 
\end{itemize} 

}

\frame{

\frametitle{Fixed point types of properties}

\begin{itemize} 
 
\item a \textit{fixed point} of a property $P$ is an $a$ such
  that $a:P(a)$

 
\item a \textit{fixed point type} of $P$ is a type such that
  anything of that type will be a fixed point of $P$.
  $\mathcal{F}(P)$ is such a fixed point type (one with the minimal
  number of required fields).  
 
\end{itemize} 

}

\frame{

\frametitle{Copredication}

\begin{itemize} 
 
\item  $\lambda r$:\smallrecord{\tfield{x}{{\it Ind\/}} \\
                           \tfield{event}{\textit{Event}}}(\smallrecord{\tfield{c$_{\mathrm{forever}}$}{take\_forever\_ev($r$.x,$r$.event)}}) 
 
\item $\lambda r$:\smallrecord{\tfield{x}{{\it Ind\/}} \\
                           \tfield{food}{\textit{Food}}}(\smallrecord{\tfield{c$_{\mathrm{delicious}}$}{be\_delicious\_fd($r$.x,$r$.food)}})

\item $\lambda r$:\smallrecord{\tfield{x}{{\it Ind\/}} \\
                           \tfield{food}{\textit{Food}}\\
                         \tfield{event}{\textit{Event}}}(\smallrecord{\tfield{c$_{\mathrm{delicious}}$}{be\_delicious\_fd($r$.x,$r$.food)}\\
\tfield{c$_{\mathrm{forever}}$}{take\_forever\_ev($r$.x,$r$.event)}})

\item $\lambda r$:\smallrecord{\tfield{x}{{\it Ind\/}}}(\record{
\tfield{food}{\textit{Food}} \\
\tfield{c$_{\mathrm{blancmange}}$}{blancmange($r$.x,food)}})

\item $\lambda r$:\smallrecord{\tfield{x}{{\it Ind\/}}}(\record{
\tfield{event}{\textit{Event}} \\
\tfield{c$_{\mathrm{game}}$}{game($r$.x,event)}})

\item $\lambda r$:\smallrecord{\smalltfield{x}{\textit{Ind}}}(\smallrecord{
\smalltfield{event}{\textit{Event}} \\
\smalltfield{food}{\textit{Food}} \\
\smalltfield{c$_{\mathrm{lunch}}$}{lunch\_ev\_fd($r$.x, event, food)}})
 
\end{itemize} 

}

\frame{

\frametitle{Dynamic interpretation of \textit{the game took forever}}

 \begin{itemize} 

 \item the($\lambda r$:\smallrecord{\tfield{x}{{\it
           Ind\/}}}(\record{
 \tfield{event}{\textit{Event}} \\
 \tfield{c$_{\mathrm{game}}$}{game\_ev($r$.x,event)}}),\\
           \hspace*{-1em}$\lambda
           r$:\smallrecord{\tfield{x}{{\it
           Ind\/}}\\
           \tfield{event}{\textit{Event}}\\
           \tfield{c$_{\mathrm{game}}$}{game\_ev(x,event)}}(\smallrecord{\tfield{c$_{\mathrm{forever}}$}{took\_forever\_ev($r$.x,$r$.event)}})) 

 \item OK -- the information passed on is a subtype of the domain type
   of the predicate 

 \end{itemize} 


 }

 \frame{

 \frametitle{Requiring coercion, lexical innovation:  \textit{the game
     was delicious}}

 \begin{itemize} 

 \item the($\lambda r$:\smallrecord{\tfield{x}{{\it
           Ind\/}}}(\smallrecord{\tfield{event}{\textit{Event}} \\
                                 \tfield{c$_{\mathrm{game}}$}{game\_ev($r$.x,event)}}),\\
           \hspace*{-1em}$\lambda
           r$:\smallrecord{\tfield{x}{{\it
           Ind\/}}\\
           \tfield{event}{\textit{Event}} \\
           \tfield{c$_{\mathrm{game}}$}{game\_ev(x,event)} \\
           \tfield{food}{\textit{Food}}}(\smallrecord{\tfield{c$_{\mathrm{delicious}}$}{be\_delicious\_fd($r$.x,$r$.food)}})) 

 \item \textit{not} OK -- the information passed on is not a subtype of
   the domain type of the predicate.  It does not provide a food
   aspect.  Something needs to be changed!

 \end{itemize} 

 }           


 \section{A counting puzzle}

 \frame{

 \frametitle{A puzzle posed by Tim Fernando}

 \begin{itemize} 

 \item Suppose that on the shelf there are
 \begin{itemize} 

 \item exactly two copies of \textit{War and Peace} 

 \item two copies of \textit{Ulysses}

 \item six copies of the Bible

 \end{itemize} 


 \item How many books are there on the shelf?
 \begin{itemize} 

 \pause \item ten (if you are counting physical objects) 

 \pause \item three (if you are counting information or textual
 objects) 

 \pause \item cf. physical vs informational objects as discussed in the
 Generative Lexicon by Pustejovsky

 \end{itemize} 


 \end{itemize} 
 }

 \frame{

 \frametitle{More complications for the Fernando puzzle}

 \begin{itemize} 

 \item Suppose one of the copies of \textit{War and Peace} is the
   Russian original and the other is a Swedish translation 

 \item Does that count as one or two books?
 \begin{itemize} 

 \pause \item Same informational content (more or less) 

 \pause \item Not the same textual content 

 \pause \item In one sense you have two copies of the same book

 \pause \item For other purposes you may want to count it as two
 distinct books

 \end{itemize} 


 \pause \item Two different editions of the Bible may contain the same
 basic text but different annotations by different biblical scholars

 \pause \item For some purposes that will count as one book and for
 others as two

 \end{itemize} 

 }

 %\section{Frames and types}

 \frame{

 \frametitle{FrameNet frame for \textit{book}}

 Frame elements:

 \begin{itemize} 

 \item Author 

 \item Components

 \item Genre

 \item Text

 \item Title

 \item Topic

 \end{itemize} 


 }

 % \frame{

 % \frametitle{Pustejovsky and Asher on dot-types}

 % \begin{itemize} 

 % \item \textit{PhysObj}

 % \item \textit{InfObj}

 % \pause \item  \textit{PhysObj}$\bullet$\textit{InfObj}

 % \pause \item $a$ : \textit{PhysObj}$\bullet$\textit{InfObj} just in case
 % it has a ``physical object aspect'' and an ``informational object
 % aspect''

 % \pause \item Asher's latest proposal in terms of category theory

 % \pause \item Why dot types and not ambiguity?
 % \begin{itemize} 

 % \item \textit{The book is important for the course but too heavy to
 %     bring to each class meeting} 

 % \item \textit{\#The bank is in good financial shape but severely
 %     eroded by the river} 

 % \end{itemize} 


 % \end{itemize} 

 % }  

 \frame{

 \frametitle{A frame type for \textit{book}}

 \begin{itemize}

 \item \record{\tfield{x}{\textit{Ind}} \\
         \tfield{physobj}{\textit{PhysObj}} \\
          \tfield{infobj}{\textit{InfObj}} \\
          \tfield{c$_{\mathrm{book}}$}{book\_ph\_inf(x,physobj,infobj)}}

 % \pause \item A record (frame, situation) is of this type just in case
 % \begin{enumerate} 

 % \pause \item it has fields with the same labels (possibly more fields) 

 % \pause \item the values in each of the fields is of the corresponding type 

 % \end{enumerate}


 \end{itemize}  


 }

 %\section{Properties and counting}

 \frame{

 \frametitle{The property \textit{is a book}}

 $\lambda r$:\smallrecord{\smalltfield{x}{\textit{Ind}}}
 (\record{\tfield{physobj}{\textit{PhysObj}} \\
          \tfield{infobj}{\textit{InfObj}} \\
          \tfield{c$_{\mathrm{book}}$}{book\_ph\_inf($r$.x,physobj,infobj)}})

 }

 % \frame{

 % \frametitle{Property extensions}

 % % \begin{itemize} 

 % % \item If $T$ is a type, then the \textit{extension} of $T$,
 % % $\down{T}$, is
 % % $\{a\mid a:T\}$ 


 % % \item If $P$ is a property, then the property-extension
 % % or \textit{P-extension} of $P$, $\downP{P}$, is  
 % % $\{a\mid\exists r[r:$\smallrecord{\smalltfield{x}{\textit{Ind}}}
 % % $\wedge\ r.\mathrm{x}=a\wedge\down{P(r)}\not=\emptyset]\}$


 % % \end{itemize} 

 % If $P$ is a property, then the property-extension
 % or \textit{P-extension} of $P$, $\downP{P}$, is  
 % $\{a\mid\exists r[r:$\smallrecord{\smalltfield{x}{\textit{Ind}}}
 % $\wedge\ \exists r'[r':P(r) 
 % \wedge\ r.\mathrm{x}=a]]\}$
 % } 

\frame{

\frametitle{Relativized property extensions}

The \textit{P-extension of} $P$ \textit{relative to label} $\ell$ (in
the body of $P$),
$\downPl{P}{\ell}$, is 
$\{a\mid\exists r[r:$\smallrecord{\smalltfield{x}{\textit{Ind}}}
$\wedge\ \exists r'[r':P(r)\wedge\ r'.\ell=a\}]]\}$ 

}   

\frame{

\frametitle{Counting books on the shelf}

\begin{itemize} 
 
\item $P$ = $\lambda r$:\smallrecord{\smalltfield{x}{\textit{Ind}}} \\
\hspace*{4em}(\record{\tfield{physobj}{\textit{PhysObj}} \\
         \tfield{infobj}{\textit{InfObj}} \\
         \tfield{c$_{\mathrm{book}}$}{book\_ph\_inf($r$.x,physobj,infobj)}
       \\ \tfield{c$_{\mathrm{on\_shelf}}$}{on\_shelf($r$.x)}}) 
 
\pause \item $\mid\downPl{P}{\mathrm{physobj}}\mid$ -- the number of books
  (physical objects) on the shelf

\pause \item $\mid\downPl{P}{\mathrm{infobj}}\mid$ -- the number of books
  (informational objects) on the shelf
 
\end{itemize} 
}  

\frame{

\frametitle{Which attributes can you use for counting?}

\begin{itemize} 
 
\item we have relativized the counting of property extensions to
  attributes in a frame
 
\pause \item  but which attributes can be used for counting?

\pause \item if we use something like the FrameNet frame corresponding
to book:
\begin{itemize} 
 
\pause \item `author' is NOT an appropriate way of counting 
 
\pause \item `title' MAY be an appropriate way of counting \pause
though Google books returns 9,950 entries for
``Introduction to Chemistry'' \ldots

\pause \item `author' and `title' together could be useful for the
majority of databases -- authors don't usually write two books with
the same title \ldots

\pause \item `text' seems like a good bet

\pause \item `physical object' is missing

\pause \item Do we mark attributes as suitable for counting or is
there something more general we can say?
 
\end{itemize} 
  
 
\end{itemize} 

} 

\frame{

\frametitle{Conclusions: Counting and identity conditions}

\begin{itemize} 
 
\item attributes which can be used for counting are the same as those
  which can be used to uniquely identify objects 
 
\pause \item  is there a general theory which will tell us for any property
  what the identity conditions of objects falling under that property are?

\pause \item we have developed a way of accounting for different ways of
  counting objects falling under a single property

\pause \item we have raised the possibility of indicating which
(sets of) attributes are ``identifying'' attributes which can be used
for counting
 
\end{itemize} 
} 

\frame{

\frametitle{Summary}

\begin{itemize} 
 
\item We showed how to analyze co-predication using record types 
 
\item We related this to the treatment of dynamic quantifiers

\item We showed how this enables to identify the need for coercion

\item We addressed a puzzle concerning the way we count objects,
  suggesting that we count objects in terms of aspects
 
\end{itemize} 

}  

\frame[allowframebreaks]{

\frametitle{References}

\nocite{Asher2011}
\nocite{Cooper2011}
\nocite{Luo2011}

\bibliographystyle{apalike}%{/Users/cooper/LaTeX/bib/mybib}
\bibliography{/Users/cooper/LaTeX/bib/bibliography}
}

  

\end{document}