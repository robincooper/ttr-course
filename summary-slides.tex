\documentclass{beamer}

\usepackage{qtree}

\usepackage{beamerthemetree}
\usepackage{color}
\setbeamertemplate{footline}[frame number]
\usepackage{graphicx}
%\usepackage[swedish]{babel} 
\usepackage[latin1]{inputenc} 
\usepackage{natbib}
 
\renewcommand{\newblock}{}
\renewcommand{\bibsection}{}

\title{Summary}
\author{Robin Cooper  \\
University of Gothenburg \\ \medskip 
Jonathan
Ginzburg \\ Univ. Paris--Diderot, Sorbonne Paris Cit�} 
\date{Type theory with records for natural language semantics, NASSLLI
2012 \\ Summary}

\AtBeginSection[]
{
   \begin{frame}[plain]
       \frametitle{Outline}
       \tableofcontents[currentsection]
   \end{frame}
}

\newcommand{\backgroundyellow}{\beamertemplateshadingbackground{yellow!40}{magenta!20}}
\newcommand{\backgroundwhite}{\beamertemplateshadingbackground{white!100}{white!100}}

\newcommand{\ignore}[1]{}

%Records

 \newcommand{\record}[1]{$\left[\mbox{\begin{tabular}{lcl} #1
 \end{tabular}}\right]$} 
%\newcommand{\record}[1]{[#1]}

\newcommand{\smallrecord}[1]{$\left[\mbox{\begin{tabular}{@{}l@{}c@{}l@{}} #1
\end{tabular}}\right]$}
\newcommand{\field}[2]{#1 & = & #2}
%\newcommand{\field}[2]{#1=#2}
\newcommand{\tfield}[2]{#1 & : & #2}
\newcommand{\smalltfield}[2]{#1:#2 & &}
%\newcommand{\tfield}[2]{#1:#2}
\newcommand{\mfield}[3]{#1=#2 & : & #3}
\newcommand{\smallmfield}[3]{#1=#2:#3 & &}
\newcommand{\hfield}[2]{{\sc #1} & & #2}

%Types

\newcommand{\down}[1]{[\ \!\check{}\ \!#1]}
\newcommand{\downP}[1]{[\downarrow\!#1]}
\newcommand{\downPl}[2]{[\downarrow_{#2}\!#1]}

\begin{document}

\frame[plain]{\titlepage}
\frame[plain]{\frametitle{Outline}\tableofcontents}

\frame{

\frametitle{Functions and unification}

\begin{itemize} 
 
\item tools of formal semantics:  functions, binding,  \ldots
 
\item tools of formal grammar:  feature structures, unification, \ldots 
 
\end{itemize} 
  

}

\frame{

\frametitle{Types and possible worlds}

\begin{itemize} 
 
\item types rather than sets of possible worlds 
 
\item finer grain for representation of content:  distinct types can
  have the same extension

\item better to have too fine a grain than too coarse a grain

\item distinction between positive and negative propositions
 
\end{itemize} 
  

}

\frame{

\frametitle{Structured objects}

\begin{itemize} 
 
\item types are structured objects, in particular record types 
 
\item if you have fine grain, you probably want structure -- otherwise
  difficult to give identity conditions

\item structure is important for semantic coordination and learning
 
\end{itemize} 
  

}

\frame{

\frametitle{Unified theory of speech events, content and dialogue}

\begin{itemize} 
 
\item TTR is a general type theory used for analysis of speech events
  (syntax), content (semantics) and dialogue 
 
\item this points towards an explanatory theory of how cognitive
  evolution led to linguistic ability

\item we gave some indication that our ability has developed for basic
  event perception and reasoning about events, shared with non-human animals
 
\end{itemize} 
  

}

\frame{

\frametitle{Some things we are currently working on}

\begin{itemize} 
 
\item probabilistic TTR 
 
\item concept acquisition by robots

\item semantic coordination and language acquisition in dialogue

\item relating TTR to neural implementation

\item implementation of TTR tools

\item semantic benchmarks

\item emotion and meaning

\item relating language and music 
 
\end{itemize} 


}  


% \frame[allowframebreaks]{

% \frametitle{References}

% \nocite{Asher2011}
% \nocite{Cooper2011}
% \nocite{Luo2011}

% \bibliographystyle{apalike}%{/Users/cooper/LaTeX/bib/mybib}
% \bibliography{/Users/cooper/LaTeX/bib/bibliography}
% }

  

\end{document}